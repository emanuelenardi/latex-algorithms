\subsection*{Visita di un grafo orientato (\texttt{visita})}

Dato un grafo orientato e un nodo di partenza \(S\), calcolate il numero di nodi raggiungibile da \(S\).

\paragraph{Formato dell'input}
La prima riga del file di input contiene tre interi, \(N\), \(M\) e \(S\). \(N\) è il numero di nodi, \(M\) il numero di archi, \(S\) il nodo di partenza.
Le successive \(M\) righe contengono \textbf{due interi per riga}, il nodo di partenza e di arrivo degli archi.

\paragraph{Formato dell'output}
L'output deve contenere un intero, uguale al numero di nodi raggiungibili da \(S\) (\(S\) incluso).

% \inputminted[firstline=7]{assets/codes/visita_bfs.cpp}
% caption = {\texttt{visita BFS}}

% \newpage
% \inputminted[firstline=7]{assets/codes/visita_dfs_iterativo.cpp}
% caption = {\texttt{visita DFS} iterativa}

% \newpage
% \inputminted[firstline=7]{assets/codes/visita_dfs_ricorsivo.cpp}
% caption = {\texttt{visita DFS} ricorsiva}

\subsection*{Sottosequenza comune massimale (\texttt{lcs})}

Definiamo come sottosequenza di una stringa una qualunque sequenza di caratteri ottenibile partendo dalla stringa di partenza, ed eliminando 0 o più caratteri e senza cambiarne l'ordine.
Il nostro obiettivo è quello di trovare, date due stringhe, la più lunga sottosequenza in comune e stamparne la lunghezza.

\paragraph{Formato dell'input}
L'input consiste di due righe, ogni riga contiene una delle due stringhe.

\paragraph{Formato dell'output}
Il file di output contiene un intero, la lunghezza della sottosequenza comune massimale.

% \inputminted[firstline=7]{assets/codes/lcs.cpp}
% caption = {\texttt{lcs}}

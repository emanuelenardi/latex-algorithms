\subsection*{Ordinamento topologico (\texttt{toporder})}

Dato un grafo orientato aciclico, stampare un suo ordinamento topologico.

\paragraph{Formato dell'input}
La prima riga del file di input contiene due interi, \(N\) e \(M\). \(N\) è il numero di nodi, \(M\) il numero di archi.
Le successive \(M\) righe contengono ognuna due interi. L’\(i\)-esima riga contiene la sorgente e la destinazione dell’\(i\)-esimo arco.

\paragraph{Formato dell'output}
La \emph{prima ed unica} riga del file di output deve contenere \(N\) interi separati da spazio, l’ordinamento topologico trovato dall’algoritmo.

% \inputminted[firstline=7]{{assets/codes/toporder_dfs.cpp}
% caption = {\texttt{toporder DFS} (ricorsivo)}

% \newpage
% \inputminted[firstline=7]{{assets/codes/toporder_it.cpp}
% caption = {\texttt{toporder} iterativo}

\subsection*{Pillole (\texttt{pillole})}

La zia Lucilla deve assumere ogni giorno mezza pillola di una certa medicina.
Lei inizia il trattamento con una bottiglia che contiene esattamente \(N\) pillole.

Durante il primo giorno lei prende una pillola dalla bottiglia, la spezza in due, ne ingerisce una metà e rimette l'altra metà nella bottiglia.

Nei giorni seguenti lei prende un pezzo a caso della bottiglia (potrebbe essere una pillola intera o una mezza pillola).
Se ha pescato una mezza pillola la ingerisce. Se ha pescato una pillola intera la spezza a metà, rimette una delle due mezze pillole nella bottiglia e ingerisce l'altra mezza pillola.

La zia può svuotare la bottiglia in tanti modi diversi.
Rappresentiamo la cura come una stringa di \(2N\) caratteri,
in cui il carattere \(i\)-esimo è "I" se la zia ha pescato una pillola intera nel giorno \(i\) e "M" se la zia ha invece pescato una mezza pillola.
Nel caso in cui la bottiglia originaria contenga 3 pillole intere, le possibili sequenze sono le seguenti:

\medskip
\begin{enumerate}
	\item \texttt{IIIMMM}
	\item \texttt{IIMIMM}
	\item \texttt{IIMMIM}
	\item \texttt{IMIIMM}
	\item \texttt{IMIMIM}
\end{enumerate}

\medskip
Il problema vi richiede di scrivere un programma che, dato \(N\), restituisca il numero di possibili sequenze nel trattamento.

\paragraph{Formato dell'input} Il file di input consiste di un unica linea contenente l’intero \(N\),
il numero di pillole presenti nella bottiglia all’inizio della cura.

\paragraph{Formato dell'output} Il file di output contiene un unico intero, il numero di diversi modi in cui la zia finisce la bottiglia.

\paragraph{Assunzioni} L'output sarà abbastanza piccolo da poter essere mantenuto dentro un \texttt{long long int}.

% \inputminted[firstline=6]{assets/codes/pillole.cpp}
% caption = {\texttt{pillole}}

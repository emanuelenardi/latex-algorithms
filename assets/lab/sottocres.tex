\subsection*{Sottoinsieme crescente di somma massima (\texttt{sottocres})}

Vi viene dato in input un array di \(N\) interi \(A_1\)\dots\(A_N\).
Per ogni elemento potete scegliere se includerlo nell'insieme soluzione, o se ignorarlo.
Se un elemento \(A_i\) fa parte dell'insieme, tutti gli elementi che lo succedono nell'array e che hanno valore minore di \(A_i\) non possono essere inclusi nell'insieme.
In altre parole, gli elementi dell’insieme, quando stampati nell'ordine in cui si trovavano nell'array, devono formare una sequenza crescente.
Vogliamo massimizzare la somma degli elementi dell’insieme.

\paragraph{Formato dell'input}
La prima riga del file di input contiene un intero \(N\). La seconda riga contiene \(N\) interi separati da spazio.

\paragraph{Formato dell'output}
Il file di output deve contenere un intero, la somma degli elementi dell'insieme di somma massima.

% \inputminted[firstline=6]{assets/codes/sottocres.cpp}
% caption = {\texttt{sottocres}}

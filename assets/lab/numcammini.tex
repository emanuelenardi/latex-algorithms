\subsection*{Numero di cammini minimi (\texttt{numcammini})}

Vi viene dato in input un grafo orientato ed una coppia di nodi \(S\) e \(T\).
Dovete trovare quanti diversi cammini di lunghezza minima ci sono fra \(S\) e \(T\).

\paragraph{Formato dell'input}
La prima riga del file di input contiene quattro interi, \(N\), \(M\), \(S\) e \(T\).
\(N\) è il numero di nodi, \(M\) il numero di archi, \(S\) il nodo di partenza e \(T\) il nodo di arrivo.
Le successive \(M\) righe contengono due interi per riga, il nodo di partenza e di arrivo degli archi.

\paragraph{Formato dell'output}
Il file di output deve contenere due interi separati da spazio.
Il primo è uguale alla distanza fra \(S\) e \(T\), il secondo al numero di percorsi di lunghezza minima da \(S\) a \(T\).

% \inputminted[firstline=8]{assets/codes/numcammini.cpp}
% caption = {\texttt{numcammini}}

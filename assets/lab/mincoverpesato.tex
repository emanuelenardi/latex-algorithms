\subsection*{Min-Cover su albero pesato (\texttt{mincoverpesato})}

Dato un albero, un sottoinsieme dei suoi nodi \(S\) è un Node-Cover se, per ogni arco dell’albero, uno dei suoi due nodi fa parte di \(S\).
Vi viene dato in input un albero con dei pesi sui nodi. Trovare il Node-Cover con il peso minimo e stamparne la dimensione.

\paragraph{Formato dell'input}
La prima riga del file di input contiene \(N\), il numero di nodi.
La riga successiva contiene \(N\) interi separati da spazio. L'\(i\)-esimo intero rappresenta il peso del nodo \(i\).
Le successive \(N - 1\) contengono ognuna una coppia \(P_i\), \(F_i\), una coppia padre-figlio.

\paragraph{Formato dell'output}
Il file di output deve contenere un intero, il peso del Node-Cover di peso minimo.

% \inputminted[firstline=6]{assets/codes/mincover_pesato.cpp}
% caption = {\texttt{mincoverpesato}}

\subsection*{Sottomatrice di somma massima (\texttt{sottomat})}

Data una matrice di interi da \(R\) righe e \(C\) colonne, trovare il rettangolo al suo interno di somma massima. Stamparne
la somma.

\paragraph{Formato dell'input}
La prima riga del file di input contiene due interi, \(R\) e \(C\), rispettivamente il numero di \textbf{righe} e di \textbf{colonne} della matrice. Le successive
\(R\) righe contengono ognuna \(C\) interi separati da spazio. L’\(i\)-esimo intero sulla \(j\)-esima riga corrisponde al valore
della matrice sulla riga \(i\) e colonna \(j\).

\paragraph{Formato dell'output}
L'output deve contenere un intero uguale alla somma della sottomatrice di somma massima.

\inputminted[
	firstline = 5,
	% caption = {\texttt{sottomatrice}}
]{cpp}{assets/codes/sottomat.cpp}

\begin{figure}[H]

\[
\begin{pmatrix}
	2 & -9 & 2 & 3 \\
	1 & 4 & 5 & 1  \\
	-2 & 3 & 4 & 1 \\
\end{pmatrix}
\Longrightarrow
\begin{pmatrix}
	2 & -7 & -5 & -2 \\
	1 & 5 & 10 & 11  \\
	-2 & 1 & 5 & 6   \\
\end{pmatrix}
\]

\caption[Rappresentazione con le matrici]{Rappresentazione della matrice di partenza (a sinistra) e di quella che contiene le somme parziali (a destra)}
\label{fig:sottomat-matrici}
\end{figure}

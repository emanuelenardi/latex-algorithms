\subsection*{Min-Cover su albero (\texttt{mincover})}

Dato un albero, un sottoinsieme dei suoi nodi \(S\) è un Node-Cover se, per ogni arco dell’albero, uno dei suoi due nodi fa parte di \(S\).
Dato un albero, trovare il Node-Cover con il minimo numero di nodi e stamparne la dimensione.

\paragraph{Formato dell'input}
La prima riga del file di input contiene \(N\), il numero di nodi.
Le successive \(N - 1\) contengono ognuna una coppia \(P_i\), \(F_i\), una coppia padre-figlio.

\paragraph{Formato dell'output}
Il file di output deve contenere un intero, la dimensione del Node-Cover

% \inputminted[firstline=6]{assets/codes/mincover.cpp}
% caption = {\texttt{mincover}}

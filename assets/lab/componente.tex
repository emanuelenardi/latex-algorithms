\subsection*{Componente fortemente connessa (\texttt{componente})}

Dato un grafo orientato, trovare la dimensione della componente fortemente connessa di dimensione massima.
Un insieme di nodi forma una componente fortemente connessa se esiste un percorso fra ogni coppia di nodi in entrambe le direzioni.

\paragraph{Formato dell'input}
La prima riga del file di input contiene due interi, \(N\) e \(M\). \(N\) è il numero di nodi, \(M\) il numero di archi.
Le successive \(M\) righe contengono ognuna due interi. L’\(i\)-esima riga contiene la sorgente e la destinazione dell’\(i\)-esimo arco.

\paragraph{Formato dell'output}
Il file di output deve contenere un intero pari alla dimensione della più grande componente fortemente connessa

% \newpage
% \inputminted[firstline=7]{cpp}{assets/codes/componente_kosaraju.cpp}
% caption = {\texttt{componente} -- Algoritmo di Kosaraju}

% \newpage
% \inputminted[firstline=7]{{cpp}{assets/codes/componente_tarjan.cpp}
% caption = {\texttt{componente} -- Algoritmo di Tarjan}

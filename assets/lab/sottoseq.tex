\subsection*{Sottosequenza di somma massima (\texttt{sottoseq})}

Data una sequenza di interi \(A[1\dots N]\), vogliamo scegliere una sottosequenza \(A[I\dots J]\) tale che la somma dei propri
elementi sia massima. Come output chiediamo la somma di tale sottosequenza.

\paragraph{Formato dell'input}
La prima riga del file di input contiene l’intero \(N\), il numero di elementi di \(A\). Le successive \(N\) righe contengono un elemento di \(A\),
da \(A_1\) a \(A_N\) .

\paragraph{Formato dell'output}
L'output deve contenere un intero uguale al valore della sottosequenza di somma massima.

\inputminted[
	firstline = 4,
	% caption = {\texttt{sottosequenza} soluzione d'esempio}
]{cpp}{assets/codes/sottoseq-es.cpp}

% caption = {\texttt{sottosequenza}}
\inputminted[firstline=4]{cpp}{assets/codes/sottoseq.cpp}

\begin{table}[!h]
	\centering
	\begin{tabular}{cllll}
		\toprule
			\texttt{cur} & \texttt{last = max(cur,cur+last);} & \texttt{last} & \texttt{mx = max(mx,last);} & \texttt{mx}\\
		\midrule
			3	&	max(3, 2)	&	3	& max(-1, 3)	& 3	 \\
			-2	&	max(-2,1)	&	1	& max(3, 1)		& 3  \\
			4	&	max(4,1)	&	4	& max(3, 4)		& 4  \\
			1	&	max(1,5)	&	5	& max(4, 5)		& 5  \\
			5	&	max(5,10)	&	10	& max(5,1 0)	& 10 \\
		\bottomrule
	\end{tabular}
	\caption{Tabella di calcolo sul secondo esempio della descrizione del problema}
	\label{tab:sottoseq-es2}
\end{table}

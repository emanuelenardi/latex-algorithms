% NOTE creare un file a parte con le macro del testo
\NewDocumentCommand\exercise{m}{%
	\begin{center}
		\fbox{#1}%
	\end{center}
}

% NOTE indicazione che quel particolare argomento potrebbe essere richiesto all'orale
\usepackage{fontawesome}
\NewDocumentCommand{\SimboloOrale}{}{%
	\faExclamationTriangle%
}
\NewDocumentCommand{\Orale}{}{%
	\normalfont\small\marginnote{%
		\SimboloOrale%
	}%
}

% NOTE parole in lingue diverse dall'italiano (inglese)
\NewDocumentCommand{\foreign}{m}{%
	\selectlanguage{english}{%
		\emph{#1}%
	}%
}

% NOTE simbolo di ricorrenza
\NewDocumentCommand{\T}{ s o m }{
	\ensuremath{%
		\IfValueTF{#2}
			{T_{#2}}
			{T}
		\left(%
		\IfBooleanTF{#1}
			{\left\lfloor#3\right\rfloor}
			{#3}
		\right)
	}
}

% NOTE per avere i parametri allineati a sx con un box di dimensione pari alla stringa più lunga
\usepackage{eqparbox}
\NewDocumentCommand{\params}{ mo }
	{ \tcp{\eqparbox{params}{\(#1\):} #2}\vspace{-3pt} }

% NOTE 🔖 [Big O and related notations](texblog.org/2014/06/24/)
\NewDocumentCommand{\Omicron}{o}{
	\IfValueTF{#1}
		{\mathcal{O}(#1)}
		{\mathcal{O}}
}
\let\oldTheta\Theta
\RenewDocumentCommand{\Theta}{o}{
	\IfValueTF{#1}
		{\oldTheta(#1)}
		{\oldTheta}
}
\let\oldOmega\Omega
\RenewDocumentCommand{\Omega}{o}{
	\IfValueTF{#1}
		{\oldOmega(#1)}
		{\oldOmega}
}

%&../preamble

% arara: pdflatex: { synctex: no }
% arara: latexmk: { clean: partial }
\ifstandalone
\begin{document}
\begin{algorithm}[H]
\fi

\prototype{\branchAndBound{\datiProblema, \Item{} \(S\), \Int \(i\), \datiParziali}}{
	\Set \(C \Assign \getChoices{\datiProblema, n, i, \datiParziali}\) \Comment*[l]{determina l'insieme in funzione delle scelte precedenti \(S[1 \dots i-1]\)}

	\BlankLine
	\tcp{esamino ogni scelta}
	\ForEach{\(c \in C\)}{
		\(S[i] \Assign c\)\;

		\BlankLine
		\tcp{calcolo il lower bound}
		\Int \(lb \Assign \lowerBound{\datiProblema, S, i, \datiParziali}\) \Comment*[l]{calcolato in base alle scelte fatte fin'ora}

		\BlankLine
		\If(\tcp*[h]{se il limite inferiore non eccede il costo minimo}){\(lb < minCost\)}{

			\BlankLine
			\uIf(\tcp*[h]{posso ancora effettuare delle scelte}){\(i < n\)}{

				\BlankLine
				\tcp{faccio ricorsivamente le scelte successive}
				\branchAndBound{\datiProblema, S, \(i+1\), \datiParziali}
				\BlankLine

			}
			\Else{

				\BlankLine
				\tcp{se il costo della soluzione trovata è minore del costo della soluzione parziale}
				\If{\(\cost(S,i) < minCost\)}{
					\tcp{allora la soluzione trovata è migliore del minimo parziale}

					\BlankLine
					\(minSol \Assign S\) \Comment*[l]{aggiorno la soluzione minima parziale}
					\(minCost \Assign \cost(S,i)\)  \Comment*[l]{aggiorno il costo minimo parziale}
				}
			}
		}

	}
}

\ifstandalone
\end{algorithm}
\end{document}
\fi

%&../preamble

% arara: pdflatex: { synctex: no }
% arara: latexmk: { clean: partial }
\ifstandalone
\begin{document}
\begin{algorithm}[H]
\fi

\prototype{\Set \greedyTsp{\Graph \(G\)}}{

	\BlankLine
	\tcp{creazione strutture dati}
	\Set \(result \Assign  \setConstructor\)\;
	\mfSet \(M \Assign  \mfConstructor{G.\setSize}\)\;
	\tcp{archi in ingresso ad un nodo}
	\Array{\Int} \(edges\) \Assign \new \Array{\Int}[1][G.\setSize] = \{ 0 \} \tcp{no.\ archi della catena}

	\BlankLine
	\(A\) \Assign \{ ordina gli archi per peso decrescente \}

	\BlankLine
	\tcp{per ogni arco che appartiene all'insieme degli archi}
	\ForEach{\((u,v) \in A\)}{
		\tcp{se non si è formato un ciclo}
		\If{\(edges[u] < 2\) \And \(edges[v] < 2\) \And \(M.\mfFind{u} \Neq M.\mfFind{v}\)}{

			\BlankLine
			\tcp{prendo nota dell'arco inserito}
			\(S.\setInsert{\((u,v)\)}\)\;

			\BlankLine
			\tcp{aggiorno il no.\ di lati entranti nei due nodi}
			\(\Increment{edges[u]}\)\;
			\(\Increment{edges[v]}\)\;

			\BlankLine
			\tcp{li considero come un unico nodo}
			\(M.\mfMerge{u,v}\)\;
		}
	}

	\BlankLine
	\tcp{chiudo il circuito}
	\Int \(u \Assign  1\)\;
	\lWhile{\(edges[1] \Neq 1\)}{
		\(\Increment{u}\)
	}

	\BlankLine
	\Int \(v \Assign  u + 1\)\;
	\lWhile{\Not \(edges[v] \Neq 1\)}{
		\(\Increment{v}\)
	}

	\BlankLine
	\tcp{chiusura del circuito hamiltoniano}
	\(S.\setInsert{\((u,v)\)}\)

	\BlankLine
	\tcp{restituisco l'insieme che archi che costituisce il percorso}
	\Return \(S\)\;
}

\ifstandalone
\end{algorithm}
\end{document}
\fi

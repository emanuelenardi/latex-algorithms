\documentclass[varwidth=6in]{standalone}
\usepackage{../_preamble}

% arara: pdflatex: { synctex: no }
% arara: latexmk: { clean: partial }
\begin{document}

\ifstandalone
\NoCaptionOfAlgo
\begin{algorithm}[H]
\caption[Visita di un grafo]{}
\fi

\prototype{\visitaGrafo{\Graph G, \Node r}}{
	\Set \(S\) \Assign \setConstructor \Comment*[l]{insieme di nodi (\Stack, \Queue)}
	\(S.\setInsert{r}\) \Comment*[l]{inserisco il nodo}

	\BlankLine
	\tcp{ho visitato il nodo}
	\{ \accentColor{marca il nodo \(r\) come \enquote{scoperto}} \}

	% \BlankLine
	% \tcp{fintanto che l'insieme non è vuoto}
	% \While{\(S.\setSize > 0\)}{
	%
	% 	\BlankLine
	% 	\tcp{la politica di rimozione dipende dal problema}
	% 	\Node \(u \Assign S.\setRemove\)\;
	%
	% 	\BlankLine
	% 	\{ \accentColor{esamina il nodo \(u\)} \}
	%
	% 	\BlankLine
	% 	\ForEach{\(u \in G.\adj{u}\)}{
	%
	% 		\BlankLine
	% 		\{ \accentColor{esamina l'arco \((u,v)\)} \}
	%
	% 		\BlankLine
	% 		\If{\(v\) non è già stato scoperto}{
	%
	% 			\BlankLine
	% 			\tcp{serve a non inserire il nodo più di una volta}
	% 			\{ \accentColor{marca il nofo \(v\) come \enquote{scoperto}} \}\;
	% 			\(S.\setInsert\) \Comment*[l]{inserisce il nodo nell'insieme}
	% 		}
	% 	}
	% }
}

\ifstandalone
\end{algorithm}
\RestoreCaptionOfAlgo
\fi

\end{document}

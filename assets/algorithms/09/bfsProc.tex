%&../preamble

% \documentclass[varwidth=6in]{standalone}
% \usepackage{../preamble}

% arara: pdflatex: { synctex: no }
% arara: latexmk: { clean: partial }
\begin{document}

\ifstandalone
\NoCaptionOfAlgo
\begin{algorithm}[H]
\caption[Procedura generica per visita in ampiezza di un grafo]{}
\fi

\tcp{visitare tutti i nodi a distanza \(k\) perima di visitare i nodi a distanza \(k+1\)}
\prototype{\bfsProc{\Graph \(G\), \Node \(r\)}}{
	\Queue \(S\) \Assign \queueConstructor\;
	\(S.\queueInsert{r}\) \Comment*[l]{inserisco la radice}

	\BlankLine
	\tcp{inizializzazione}
	\Array{\Bool} \(visitato\) \Assign \Array{\Bool}{1}{G.n}\;
	\lForEach{\(u \in G.\VV - \{r\}\)}{
		\ArrayCall{visitato}{u} \Assign \False\;
	}
	\ArrayCall{visitato}{r} \Assign \True \Comment*[l]{radice visitata}

	\BlankLine
	\tcp{visita del grafo}
	\While{\Not \(S.\setEmpty\)}{
		\Node \(u\) \Assign \(S\).\queueRemove\;

		\BlankLine
		\{ \accentColor{esamina il nodo \(u\)} \}

		\BlankLine
		\ForEach{\(u \in G.\adj{u}\)}{

			\BlankLine
			\{ \accentColor{esamina l'arco \((u,v)\)} \}

			\BlankLine
			\If{\Not \ArrayCall{visitato}{v}}{

				\BlankLine
				\ArrayCall{visitato}{v} \Assign \True\;
				\(S.\queueInsert\)\;
			}
		}
	}
}

\ifstandalone
\end{algorithm}
\RestoreCaptionOfAlgo
\fi

\end{document}

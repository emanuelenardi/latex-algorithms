%&../preamble

% \documentclass[varwidth=6in]{standalone}
% \usepackage{../preamble}

% arara: pdflatex: { synctex: no }
% arara: latexmk: { clean: partial }
\begin{document}

\ifstandalone
\NoCaptionOfAlgo
\begin{algorithm}[H]
\caption[]{}
\fi

\tcp{applicabile solo ai DAG, in quanto non hanno archi all'indietro}
\prototype{\Bool \hasCycle{\Graph \(G\), \Node \(u\)}}{
	\tcp{\(u\): il primo nodo che viene visitato}

	\BlankLine
	\Increment{time}\;
	\(dt[u]\) \Assign \(time\) \Comment*[l]{tempo di scoperta}

	\BlankLine
	\ForEach{\(u \in G.\adj{u}\)}{

		\BlankLine
		\uIf{\(dt[v] \Equal 0\)}{
			\tcp{non ho ancora scoperto questo nodo}

			\BlankLine
			\tcp{effettuo una visita ricorsiva}
			\If{\hasCycle{G,v}}{
				\Return \True\;
			}

		}
		\BlankLine
		\tcp{logica dell'algoritmo}
		\ElseIf{\(dt[u] < dt[v]\) \And \(ft[v] \Neq 0\)}{
			\tcp{se raggiungo un mio discendente e non ho ancora terminato la mia visita, allora ho trovato un arco \emph{all'indietro} e quindi un ciclo}
			\Return \True\;
		}
	}

	\BlankLine
	\Increment{time}\;
	\(ft[u]\) \Assign \(time\) \Comment*[l]{tempo di fine}

	\BlankLine
	\tcp{non ho trovato un ciclo}
	\Return \False\;
}

\ifstandalone
\end{algorithm}
\RestoreCaptionOfAlgo
\fi

\end{document}

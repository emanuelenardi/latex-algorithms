%&../preamble

% \documentclass[varwidth=6in]{standalone}
% \usepackage{../preamble}

% arara: pdflatex: { synctex: no }
% arara: latexmk: { clean: partial }
\ifstandalone
\begin{document}
\NoCaptionOfAlgo
\begin{algorithm}[H]
\caption[Approccio ingordo al problema del resto]{}
\fi

\prototype{\restoGreedy{\Array{\Int} \(t\), \Int \(n\), \Int \(R\), \Array{\Int} \(x\)}}{

	\{ Ordina le monete in modo \emph{decrescente} \}\;
	\tcp{\(\Omicron(n)\) se già ordinato, \(\Omicron(n \log n)\) altrimenti}

	\BlankLine
	\From(\Comment*[h]{\(\Omicron(n)\)}){\(i \Assign 1\) \DownTo \(n\)} {

		\BlankLine
		\tcp{il numero di monete di taglio massimo}
		\(x[i] \Assign \left\lfloor \dfrac{R}{t[i]} \right\rfloor\)\;

		\BlankLine
		\tcp{calcolo il resto rimanente}
		\(R \Assign R - x[i] \cdot t[i]\)\;
	}

	\Return \(R\)
}

\ifstandalone
\end{algorithm}
\RestoreCaptionOfAlgo
\end{document}
\fi

divido il resto per il taglio della moneta più grande trovando il numero di monete massimo di quel taglio, dopodiché
tolgo il valore della somma quelle monete dal resto.

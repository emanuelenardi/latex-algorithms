%&../preamble

% arara: pdflatex: { synctex: no }
% arara: latexmk: { clean: partial }

\SetKwFunction{diameter}{diametro}

\ifstandalone
\begin{document}
\begin{algorithm}[H]
\caption[Calcola il quadrato di un grafo orientato]{}
\fi

Si effettuano \(n\) visite in ampiezza, una a partie da ogni nodo, e si memorizza il massimo valore di distanza trovato in una visita.

\BlankLine
\tcp{calcola il \enquote{più lungo cammino breve}}
\prototype{\diameter{\Graph G}}{
	\Int \(max \Assign 0\)\;
	\Queue \(S\) \Assign \new \queueConstructor\;
	\(S.\queueInsert{r}\)\;

	\BlankLine
	\tcp{inizializzazione}
	\Array{\Bool} \(dist\) \Assign \new \Array{\Bool}{1}{G.\graphSize}\;
	\ForEach{\(u \in G.\VV - \{r\}\)}{
		\(dist[u] \Assign -1\)\;
	}

	\BlankLine
	\(dist[r] = 0\) \Comment*[l]{la radice dista 0 da sé stessa}
	\While{\Not \(S.\queueEmpty\)}{
		\Node \(u \Assign S.\queueRemove\) \Comment*[l]{estrai il nodo}

		\BlankLine
		\ForEach{\(v \in G.\adj{u}\)}{

			\BlankLine
			\If{\(dist[v] < 0\)}{
				\tcp{non ho ancora scoperto il nodo}

				\BlankLine
				\(dist[v] \Assign dist[u] + 1\) \Comment*[l]{imposta la distanza}
				\If{\(dist[v] > max\)}{
					\(max = dist[v]\) \Comment*[l]{aggiorna il \(max\)}
				}

				\BlankLine
				\(S.\queueInsert{v}\) \Comment*[l]{inserisci il nodo}
			}
		}
	}
}

\ifstandalone
\end{algorithm}
\end{document}
\fi

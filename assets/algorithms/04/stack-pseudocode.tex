%&../preamble

% \documentclass[varwidth=6in]{standalone}
% \usepackage{../preamble}

% arara: pdflatex: { synctex: no }
% arara: latexmk: { clean: partial }
\begin{document}

% \ifstandalone
\NoCaptionOfAlgo
\algorithmstyle{ruled}
\begin{algorithm}[H]
\caption{Struttura dati pila basata su vettore in pseudocodice}
% \fi
\begin{multicols}{2}

\BlankLine
\Item{} \(A\)	\Comment*[r]{elementi}
\Int \(n\)		\Comment*[r]{cursore}
\Int \(m\)		\Comment*[r]{dimesione massima}

\BlankLine
\tcp{crea una pila vuota}
\prototype{\Stack \stackConstructor{\Int dim}}{
	\Stack \(t =\) \new \Stack\;
	\(t.A =\) \new \Array{\Int}{0}{dim-1}\;
	\(t.m = dim\)\;
	\(t.n = 0\)\;

	\BlankLine
	\Return \(t\)\;
}

\BlankLine
\tcp{leggi l'elemento in cima alla pila}
\prototype{\Item \stackTop}{
	\precondition{\(n > 0\)}

	\BlankLine
	\Return \ArrayCall{A}{n}\;
}

\vspace{10pt}

% \BlankLine
\tcp{restituisce \True se la pila è vuota}
\prototype{\Bool \stackEmpty}{
	\Return \(n \Equal 0\)
}

\BlankLine
\tcp{estrae l'elemento in cima alla pila e lo restituisce al chiamante}
\prototype{\Item \stackPop}{
	\precondition{\(n > 0\)}

	\BlankLine
	\Item \(t = \ArrayCall{A}{n}\)\;
	\Decrement{n}\;

	\BlankLine
	\Return \(t\)
}

\BlankLine
\tcp{inserisce \(v\) in cima alla pila}
\prototype{\stackPush{\Item v}}{
	\precondition{\(n < m\)}

	\BlankLine
	\Increment{n}\;
	\(\ArrayCall{A}{n} = v\)
}

\vphantom{0pt}

\end{multicols}
\ifstandalone
\end{algorithm}
\algorithmstyle{tworuled}
\RestoreCaptionOfAlgo
\fi

\end{document}

%&../preamble

% \documentclass[varwidth=6in]{standalone}
% \usepackage{../preamble}

% arara: pdflatex: { synctex: no }
% arara: latexmk: { clean: partial }
\begin{document}

\begin{minipage}{\linewidth}
\NoCaptionOfAlgo
\algorithmstyle{ruled}
\begin{algorithm}[H]
\caption{Specifica \textsc{Sequence}}
\begin{multicols}{2}

Una struttura dati \emph{dinamica}, \emph{lineare} che rappresenta una sequenza \emph{ordinata} di valori, dove lo stesso valore può comparire più volte.

\BlankLine
\sequenceConstructor

\BlankLine
\tcp{INTERPRETARE}
\Bool \listEmpty \Comment*[r]{\True se la sequenza è vuota}
\Bool \listEnd \Comment*[r]{\True se \(p\) è uguale a \(pos_0\) o a \(pos_{n+1}\)}

\BlankLine
\tcp{LEGGERE}
\Postype \listHead \Comment*[r]{posizione del primo elemento}
\Postype \listTail \Comment*[r]{posizione dell'ultimo elemento}

\BlankLine
\tcp{ITERARE}
\Postype \listSucc \Comment*[r]{posizione dell'elem.\ che segue \(p\)}
\Postype \listPred \Comment*[r]{posizione dell'elem.\ che precede \(p\)}

\vspace{20pt}

\BlankLine
\tcp{MODIFICA}

\BlankLine
\tcp{inserisce l'elemento di tipo \Item nella posizione \(p\),}
\tcp{ritorna la nuova posizione,}
\tcp{che diviene il predecessore di \(p\)}
\Postype \listInsert{\Postype p, \Item v}

\tcp{rimuove l'elemento contenuto nella pos.\ \(p\),}
\tcp{ritorna il successore di \(p\),}
\tcp{che diviene il predecessore di \(p\)}
\Postype \listRemove{\Pos p}

\BlankLine
\tcp{legge l'elemento di tipo \Item}
\tcp{contenuto nella posizione \(p\)}
\listRead{\Postype p}

\BlankLine
\tcp{scrive l'elemento \(v\) di tipo \Item}
\tcp{nella posizione \(p\)}
\listWrite{\Postype p, \Item v}

\vphantom{0pt}
\end{multicols}
\end{algorithm}
\algorithmstyle{tworuled}
\RestoreCaptionOfAlgo
\end{minipage}

\end{document}

\NoCaptionOfAlgo
\begin{algorithm}
\caption[Sequenza]{}
\begin{multicols}{2}

\BlankLine
\tcp{restituisce true se è vuota}
\Bool \listEmpty{}

\BlankLine
\tcp{restituisce true se \(p\) è uguale a \({pos}_0\) oppure a \({pos}_{n+1}\)}
\Bool \listEnd{\Postype \(p\)}

\BlankLine
\tcp{restituisce la posizione del primo elemento}
\Postype \listHead{}

\BlankLine
\tcp{restituisce la posizione dell'ultimo elemento}
\Postype \listTail{}

\BlankLine
\tcp{restituisce la posizione dell'elemento che segue \(p\)}
\Postype \listSucc{\Postype \(p\)}

\BlankLine
\tcp{restituisce la posizione dell'elemento che precede \(p\)}
\Postype \listPred{\Postype \(p\)}

\vspace{15pt}

\BlankLine
\tcp{inserisce l'elemento \(v\) di tipo \Item nella posizione \(p\)}
\tcp{restituisce la posizione del nuovo elemento, che diviene il predecessore di \(p\)}
\Pos \listInsert{\Postype \(p\), \Item \(v\)}

\BlankLine
\tcp{rimuove l'elemento contenuto nella posizione \(p\)}
\tcp{restituisce la posizione del successore di \(p\), che diviene il successore del predecessore di \(p\)}
\Pos \listRemove{\Postype \(p\)}

\BlankLine
\tcp{legge l'elemento di tipo \Item contenuto nella posizione \(p\)}
\Item listRead{\Postype \(p\)}

\BlankLine
\tcp{scrive l'elemento \(v\) di tipo \Item nella posizione \(p\)}
\listWrite{\Postype \(p\), \Item \(v\)}

\vphantom{0pt}
\end{multicols}
\end{algorithm}
\RestoreCaptionOfAlgo

%&../preamble

% \documentclass[varwidth=6in]{standalone}
% \usepackage{../preamble}

% arara: pdflatex: { synctex: no }
% arara: latexmk: { clean: partial }
\begin{document}

\ifstandalone
\NoCaptionOfAlgo
\algorithmstyle{ruled}
\begin{algorithm}[H]
\caption{Specifica \textsc{Dictionary}}
\fi

Un dizionario è una struttura dati che rappresenta il concetto matematico di \emph{relazione univoca} o associazione chiave-valore.

\BlankLine
\dictionaryConstructor

\BlankLine
\Item \dictLookup{\Item \(k\)} \Comment*[r]{rest. il valore associato alla chiave \(k\), \Nil altrimenti}
\Item \dictInsert{\Item \(k\), \Item \(v\)} \Comment*[r]{associa il valore \(v\) alla chiave \(k\)}
\dictRemove{\Item \(k\)} \Comment*[r]{rimuove l'associazione della chiave \(k\)}

\ifstandalone
\end{algorithm}
\algorithmstyle{tworuled}
\RestoreCaptionOfAlgo
\fi

\end{document}

\NoCaptionOfAlgo
\begin{algorithm}[H]
\caption[Struttura dati dizionario]{}

Struttura dati \emph{dinamica}, \emph{non lineare} che memorizza una collezione non ordinata di elementi senza valori ripetuti.
Rappresenta i concetto matematico di \emph{relazione univoca} \(R\colon D \to C\), o associazione chiave-valore. Dove \(D\) rappresenta il dominio di elementi detti \emph{chiave}, mentre \(C\) rappresenta il codominio degli elementi detti \emph{valori}.
Ogni \emph{valore} può essere associato a più \emph{chiavi}, ma non il contrario.

\BlankLine
\Item \dictLookup{\Item k} \Comment*[r]{restituisce il valore associato alla chiave \(k\) se presente, \Nil altrimenti}

\dictInsert{\Item k, \Item v} \Comment*[r]{associa il valore \(v\) alla chiave \(k\), sovrascrive se già presente}

\dictRemove{\Item k} \Comment*[r]{rimuove l'associazione della chiave \(k\)}

\end{algorithm}
\RestoreCaptionOfAlgo

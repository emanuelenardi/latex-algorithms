\documentclass[varwidth=6in]{standalone}
\usepackage{../_preamble}

% arara: pdflatex: { synctex: no }
% arara: latexmk: { clean: partial }
\begin{document}

\ifstandalone
\NoCaptionOfAlgo
\begin{minipage}{\linewidth}
\begin{algorithm}[H]
\caption[Ordinamento per far vincere l'Italia ai mondiali]{}

\begin{multicols}{2}

\BlankLine
\tcp{determina un ordinamento in cui l'Italia può vincere i mondiali}
\fi

\prototype{\Bool \canItalyWin{\Graph G, \Int s}}{
	\Array{\Int} \(visited\) \Assign \new \Array{\Int}{1}{G.\setSize}\;

	\BlankLine
	\tcp{inizializzo i nodi come non visitati}
	\ForEach{\(u \in G.\VV\)}{
		\(visited[u] \Assign 0\)\;
	}

	\BlankLine
	\tcp{identifica le componenti connesse}
	\ccdfs{G, d, 1, \(visited\)}\;

	\BlankLine
	\Return \(\sumFunction{\(visited\), 1, \(G.\setSize\)} \Equal G.\setSize\)\;
}

\ifstandalone
\begin{minipage}{.85\linewidth}
\begin{figure}\centering
	% TODO disegnare ordine topologico
\end{figure}
\columnbreak
\end{minipage}
\fi

\prototype{\printOrder{\Graph G, \Int s}}{
	\Array{\Int} \(visited\) \Assign \new \Array{\Int}{1}{G.\setSize}\;

	\tcp{inizializzo i nodi come falsi}
	\ForEach{\(u \in G.\VV\)}{
		\(visited[u] \Assign 0\)\;
	}

	\Return \(\printOrderRec{\(visited\), 1, \(G.\setSize\)} \Equal G.\setSize\)\;
}

\prototype{\printOrderRec{\Graph G, \Node u, \Array{\Bool} \(visited\)}}{
	\(visited[u] \Assign \True\)\;

	\BlankLine
	\ForEach{\(v \in G.\adj{u}\)}{
		\printOrderRec{G, v, \(visited\)}\;
		\Print \((u,v)\)\;
	}
}

\ifstandalone
\end{multicols}
\end{algorithm}
\end{minipage}
\RestoreCaptionOfAlgo
\fi

\end{document}

\documentclass[varwidth=6in]{standalone}
\usepackage{../preamble}

% arara: pdflatex: { synctex: no }
% arara: latexmk: { clean: partial }
\begin{document}

\ifstandalone
\NoCaptionOfAlgo
\begin{algorithm}[H]
\caption[Bilanciamento dell'albero tramite rotazione a sinistra]{}
\fi

\BlankLine
\tcp{Inserimento di un nodo in un albero Red-Black}
\prototype{\insertNode{\Tree T, \Tree k, \Item x}}{
	\Tree \(p \Assign \Nil\) \Comment*[l]{riferimento al padre}
	\Tree \(u \Assign T\) \Comment*[l]{riferimento alla radice}

	\BlankLine
	\tcp{cerco posizione inserimento}
	\While{\(u \Neq \Nil\) \And \(u.\varKey \Neq k\)}{
		\(p \Assign u\)\;
		\(u\) \Assign \iif{\(k < u.\varKey, u.\varLeft, u.\varRight\)}\;
	}

	\BlankLine
	\eSea{\(u \Neq \Nil\) \And \(u.\varKey \Equal k\)}{
		\tcp{la chiave è già presente, aggiorno il valore}

		\BlankLine
		\(u.\varValue \Assign v\)\;
	}{
		\tcp{la chiave non è presente}
		\tcp{creo un nodo coppia chiave-valore}
		\Tree \(new\) \Assign \treeConstructor{\(k, v\)}\;

		\BlankLine
		\tcp{collego il nodo creato}
		\shortcut{\(p, new, k\)}\;

		\alert{\treeFixInsert{\(new\)}}\;

		\BlankLine
		\If{\(p \Equal \Nil\)}{
			\(T \Assign new\) \Comment*[l]{primo nodo ad essere inserito}
		}
	}

	\BlankLine
	\tcp{restituisco l'albero non modificato o il nuovo nodo}
	\Return \(T\)\;
}

\ifstandalone
\end{algorithm}
\RestoreCaptionOfAlgo
\fi

\end{document}

%&../preamble

% arara: pdflatex: { synctex: no }
% arara: latexmk: { clean: partial }
\ifstandalone
\begin{document}
\begin{algorithm}[H]
\fi

\BlankLine
\tcp{CONTENUTO DI UN NODO}

\BlankLine
\Tree \varParent\;
\Tree \varLeft\;
\Tree \varRight\;
\alert{\Int \varColor} \Comment*[l]{\RED o \BLACK}
\Tree \varKey\;
\Tree \varValue\;

% \BlankLine
% \tcp{GETTERS}
%
% \BlankLine
% \Item \treeKey\;
% \Item \treeValue\;
% \Tree \treeLeft\;
% \Tree \treeRight\;
% \Tree \treeParent\;
%
% \ifstandalone
% % \columnbreak
% \vspace{15pt}
% \vphantom{0pt}
% \fi
%
% \BlankLine
% \tcp{ORDINAMENTO}
%
% \BlankLine
% \Tree \succNode{\Tree \(t\)}\;
% \Tree \predNode{\Tree \(t\)}\;
% \Tree \minNode\;
% \Tree \maxNode\;
%
% \BlankLine
% \tcp{FUNZIONI DIZIONARIO}
%
% \BlankLine
% \Item \dictLookup{\Item \(k\)}\;
% \dictInsert{\Item \(k\), \Item \(v\)}\;
% \dictRemove{\Item \(k\)}\;
%
% \BlankLine
% \tcp{FUNZIONI INTERNE}
%
% \BlankLine
% \Item \lookupNode\;
% \insertNode{\Tree \(T\), \Item \(k\), \Item \(v\)}\;
% \removeNode{\Tree \(T\), \Item \(k\)}\;

\ifstandalone
\end{algorithm}
\end{document}
\fi

%&../preamble

% arara: pdflatex: { synctex: no }
% arara: latexmk: { clean: partial }
\ifstandalone
\begin{document}
\begin{algorithm}[H]
\caption[Insieme indipendente di intervalli disgiunti]{}
\fi

\prototype{\Set \maxInterval{\Int a, \Int b, \Int w, \Int n}}{

	\{ ordina gli intervalli per estremi di fine crescenti \}\;

	\BlankLine
	\Matrix{\Int} \(pred\) \Assign \computePredecessor{a, b, n}\;
	\Matrix{\Int} \(DP\) \Assign \new \Array{\Int}{0}{n}\;

	\BlankLine
	\tcp{riempio la tabella}
	\(DP[0] \Assign 0\)\;
	\From{\(i \Assign 1\) \DownTo \(n\)}{
		\(DP[i]\) \Assign \MathMax{\(DP[i-1]\), \(w[i] + DP[pred[i]]\)}\;
	}

	\tcp{costruisco l'insieme dei predecessori}
	\(i \Assign n\)\;
	\Set \(S \Assign \setConstructor\)\;

	\BlankLine
	\While(\Comment*[h]{fintanto che ci sono intervalli disponibili}){\( i > 0 \)}{

		\BlankLine
		\eIf(\Comment*[h]{commento}){\(DP[i-1] > \weight{i} + DP[pred[i]]\)}{
			\(\Decrement{i}\) \Comment*[h]{non considerarlo}\;
		}{
			\(S.\setInsert(i)\) \Comment*[h]{inseriscilo nell'insieme}\;
			\(i \Assign pred[i]\) \Comment*[h]{scorri gli interalli}\;
		}
	}

	\BlankLine
	\Return \(S\) \Comment*[h]{ritorna l'insieme di intervalli ordinati}\;
}

\ifstandalone
\end{algorithm}
\end{document}
\fi

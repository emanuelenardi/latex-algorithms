%&../preamble

% \documentclass[varwidth=6in]{standalone}
% \usepackage{../preamble}

% arara: pdflatex: { synctex: no }
% arara: latexmk: { clean: partial }
\begin{document}

\ifstandalone
\NoCaptionOfAlgo
\begin{algorithm}[H]
\caption[Stampa espressioni con operatori binari]{}
\begin{multicols}{2}
\fi

\prototype{\bfsProc{\Tree \(t\)}}{
	\Queue \(Q \Assign\) \queueConstructor\;
	\(Q.\queueInsert{t}\)\; \Comment*[l]{inserisci la radice}

	\BlankLine
	\While{\Not \(Q.\queueEmpty\)}{
		\tcp{fintanto che la coda non è vuota}
		\tcp{estraggo un nodo dalla coda}
		\Tree \(u \Assign Q.\queueRemove\)\;

		\BlankLine
		\tcp{visita per livelli del nodo \(u\)}
		\Print \(u\)\;

		\BlankLine
		\tcp{fintanto che ho almeno un figlio}
		\(u \Assign u.\treeChild\)\;
		\While{\(u \Neq \Nil\)}{

			\BlankLine
			\tcp{metto in coda il figlio}
			\(Q.\queueRemove{u}\)\;
			\tcp{passo al figlio destro}
			\(u \Assign u.\treeChild\)\;
		}
	}
}

\ifstandalone
% \fbox{
\begin{minipage}[t][7.5cm]{.85\linewidth}
	\vfill\centering
	\begin{center}
		\begin{forest} circled, wide
		[\textsc{a}[\textsc{b}[\textsc{c}][\textsc{d}]][\textsc{e}[\textsc{f}][\textsc{g}]]]
		\end{forest}
	\end{center}
	Sequenza: {\scshape a b e c d f g}
	\vfill\null
\end{minipage}
% }
\fi

\ifstandalone
\end{multicols}
\end{algorithm}
\RestoreCaptionOfAlgo
\fi

\end{document}

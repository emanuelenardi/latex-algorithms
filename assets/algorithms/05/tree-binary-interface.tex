%&../preamble

% \documentclass[varwidth=6in]{standalone}
% \usepackage{../preamble}

% arara: pdflatex: { synctex: no }
% arara: latexmk: { clean: partial }
\begin{document}

\ifstandalone
\NoCaptionOfAlgo
\algorithmstyle{ruled}
\begin{algorithm}[H]
\caption{Specifica \textsc{Binary Tree}}
\fi

\BlankLine
\tcp{GESTIONE ALBERO}

\BlankLine
\treeConstructor{\Item \(v\)} \Comment*[l]{costruisce un nuovo nodo, contenente \(v\), senza figli o genitori}
\Item \treeRead \Comment*[l]{legge il valore memorizzato nel nodo}
\treeWrite{\Item \(v\)} \Comment*[l]{modifica il valore memorizzato nel nodo}
\Tree \treeParent \Comment*[l]{restituisce il padre, oppure \Nil se questo nodo è radice}

\BlankLine
\BlankLine
\tcp{GESTIONE STRUTTURA}

\BlankLine
\tcp{restituiscono il figlio sinistro (destro) di questo nodo,}
\tcp{restituisce \Nil se assente}
\Tree \treeLeft\;
\Tree \treeRight\;

\BlankLine
\tcp{inserisce il sottoalbero radicato in \(t\)}
\tcp{come figlio sinistro (destro) di questo nodo}
\insertLeft{\Tree \(t\)}\;
\insertRight{\Tree \(t\)}\;

\BlankLine
\tcp{distrugge (ricorsivamente) il figlio sinistro (destro) di questo nodo}
\deleteLeft\;
\deleteRight\;

\ifstandalone
\end{algorithm}
\algorithmstyle{tworuled}
\RestoreCaptionOfAlgo
\fi

\end{document}

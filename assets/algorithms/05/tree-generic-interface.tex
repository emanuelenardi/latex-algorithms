%&../preamble

% \documentclass[varwidth=6in]{standalone}
% \usepackage{../preamble}

% arara: pdflatex: { synctex: no }
% arara: latexmk: { clean: partial }
\begin{document}

\ifstandalone
\NoCaptionOfAlgo
\algorithmstyle{ruled}
\begin{algorithm}[H]
\caption{Specifica \textsc{Generic Tree}}
\fi

\BlankLine
\tcp{GESTIONE ALBERO}

\BlankLine
\treeConstructor{\Item \(v\)} \Comment*[l]{costruisce un nuovo nodo, contenente \(v\), senza figli o genitori}
\Item \treeRead \Comment*[l]{legge il valore memorizzato nel nodo}
\treeWrite{\Item \(v\)} \Comment*[l]{modifica il valore memorizzato nel nodo}
\Tree \treeParent \Comment*[l]{restituisce il padre, oppure \Nil se questo nodo è radice}

\BlankLine
\BlankLine
\tcp{GESTIONE STRUTTURA}

\begin{minipage}{.9\textwidth}%
\begin{multicols}{2}%

\BlankLine
\tcp{restituiscono il primo figlio,}
\tcp{oppure \Nil se questo nodo è una foglia}
\Tree \treeChild\;

\BlankLine
\tcp{restituisce il prossimo fratello,}
\tcp{oppure \Nil se assente}
\Tree \treeSibling\;

\BlankLine
\tcp{inserisce il sottoalbero \(t\)}
\tcp{come primo nodo di questo nodo}
\insertChild{\Tree \(t\)}

\BlankLine
\tcp{inserisce il sottoalbero \(t\)}
\tcp{come primo figlio di questo nodo}
\insertSibling{\Tree \(t\)}

\BlankLine
\tcp{distuggi l'albero radicato}
\tcp{identificato dal primo figlio}
\deleteChild

\BlankLine
\tcp{distuggi l'albero radicato}
\tcp{identificato dal primo figlio}
\deleteSibling

\end{multicols}
\end{minipage}
\vspace{5pt}

\ifstandalone
\end{algorithm}
\algorithmstyle{tworuled}
\RestoreCaptionOfAlgo
\fi

\end{document}

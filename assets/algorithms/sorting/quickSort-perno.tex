%&../preamble

% arara: pdflatex: { synctex: no }
% arara: latexmk: { clean: partial }
\ifstandalone
\begin{document}
\begin{algorithm}[H]
\fi

\BlankLine
\tcp{sposta gli elementi più piccoli a sinistra del perno, i più grandi a destra}
\prototype{\Int \pivot{\Item{} A, \Int primo, \Int ultimo}}{
	\Item \(x \Assign A[primo]\) \Comment*[l]{il perno è il primo elemento}
	\Int \(j \Assign primo\) \Comment*[l]{il cursore parte dal primo elemento}

	\BlankLine
	\tcp{spostamenti \enquote{in-place}}
	\From{\(i \Assign primo\) \DownTo \(ultimo\)}{
		\If(\tcp*[h]{l'elemento è più piccolo del perno}){\(A[i] < x\)}{
			\Increment{j} \Comment*[l]{sposta il cursore \(j\)}
			\Swap{\(A[i]\)}{\(A[j]\)} \Comment*[l]{scambia gli elementi: \(i \leftrightarrow j\)}
		}
	}

	\BlankLine
	\tcc{a questo punto tutti gli elementi posizionati prima della posizione \(j\) sono più piccoli del perno, rimane solo da riposizionare il perno nella sua posizione finale (è ordinato)}

	\BlankLine
	\tcp{riposiziono il perno}
	\(A[primo] \Assign A[j]\)\;
	\(A[j] \Assign x\)\;

	\BlankLine
	\tcp{restituisco la posizione del perno}
	\Return \(j\)\;
}

\ifstandalone
\end{algorithm}
\end{document}
\fi

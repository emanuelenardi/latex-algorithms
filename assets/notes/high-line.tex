La High Line è un parco lineare di New York realizzato su una sezione in disuso della ferrovia sopraelevata chiamata West Side Line.

\begin{itemize}
	\item \'E un rettilineo lungo \(L\) metri corredati da aiuole e piante.
	Lungo il parco, esistono \(n\) irrigatori.
	\item L'irrigatore \(i\)-esimo è collocato ad una distanza \(D[i]\) dall'inizio della High Line.
\end{itemize}

Scheduling

Siano dati \(n\) job da sottometere ad un processore, il job \(i\)-esimo (\(1 \leqslant i \leqslant n\))
\begin{itemize}
	\item ha un deadline positiva intera \(D[i]\);
	\item un qualunque positivo intero \(G[i]\);
	\item un tempo di esecuzione pari a \(1\) (uguale per tutti).
\end{itemize}

Se il job \(i\) è eseguito entro l'istante \(D[i]\) darà un guadagno \(G[i]\), altrimenti il guadagno è \(0\).
Trovare una sequenza di esecuzione che massimizzi il guadagno

\NoCaptionOfAlgo
\begin{algorithm}[H]
\caption[]{}

\SetKwFunction{maxgain}{maxgain}

\prototype{\maxgain{\Array{\Int} \(D\), \Array{\Int} \(G\), \Int \(n\)}}{
	\{ ordina i vettori \(D\), \(G\) per guadagno decrescente \}\;
	\Int \(t \Assign 1\)\;
	\From{\(i \Assign 1\) \DownTo \(n\)}{
		\If{\(t \leqslant D[i]\)}{
			\Print \(i\)\;
			\Increment{t}\;
		}
	}
}

\end{algorithm}
\RestoreCaptionOfAlgo

Provare che l'algoritmo proposto non è corretto.

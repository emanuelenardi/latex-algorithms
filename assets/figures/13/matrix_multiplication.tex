\documentclass[border={5pt 15pt 15pt 5pt}]{standalone}

\usepackage[utf8]{inputenc}
\usepackage[upright]{fourier}

\usepackage{tikz}
\usetikzlibrary{
	,arrows
	,decorations.pathmorphing
	,matrix
}

\usepackage{xcolor}
\colorlet{brighter}{blue!50!white}
\colorlet{darker}{blue!50!black}

% arara: pdflatex: { draft: yes }
% arara: pdflatex: { synctex: no }
% arara: latexmk:  { clean: partial }
\begin{document}

% l' unite
\newcommand{\myunit}{1 cm}
\tikzset{
    node style sp/.style={draw,circle,minimum size=\myunit},
    node style ge/.style={circle,minimum size=\myunit},
    arrow style mul/.style={draw,sloped,midway,fill=white},
    arrow style plus/.style={midway,sloped,fill=white},
}

\begin{tikzpicture}[>=latex]
% les matrices
\matrix (A) [matrix of math nodes,%
             nodes = {node style ge},%
             left delimiter  = (,%
             right delimiter = )] at (0,0)
{%
  a_{11} & a_{12} & \ldots & a_{1p}  \\
  |[node style sp]| a_{21}%
         & |[node style sp]| a_{22}%
                  & \ldots%
                           & |[node style sp]| {a_{2p}}; \\
  \vdots & \vdots & \ddots & \vdots  \\
  a_{n1} & a_{n2} & \ldots & a_{np}  \\
};
\node [draw,below=10pt] at (A.south)
    { \(A\) : \textcolor{darker}{\(n\) righe} \(p\) colonne};

\matrix (B) [matrix of math nodes,%
             nodes = {node style ge},%
             left delimiter  = (,%
             right delimiter =)] at (6*\myunit,6*\myunit)
{%
  b_{11} & |[node style sp]| b_{12}%
                  & \ldots & b_{1q}  \\
  b_{21} & |[node style sp]| b_{22}%
                  & \ldots & b_{2q}  \\
  \vdots & \vdots & \ddots & \vdots  \\
  b_{p1} & |[node style sp]| b_{p2}%
                  & \ldots & b_{pq}  \\
};
\node [draw,above=10pt] at (B.north)
    { \(B\) : \(p\) righe \textcolor{darker}{\(q\) colonne}};
% matrice résultat
\matrix (C) [matrix of math nodes,%
             nodes = {node style ge},%
             left delimiter  = (,%
             right delimiter = )] at (6*\myunit,0)
{%
  c_{11} & c_{12} & \ldots & c_{1q} \\
  c_{21} & |[node style sp,darker]| c_{22}%
                  & \ldots & c_{2q} \\
  \vdots & \vdots & \ddots & \vdots \\
  c_{n1} & c_{n2} & \ldots & c_{nq} \\
};
% les fleches
\draw[brighter] (A-2-1.north) -- (C-2-2.north);
\draw[brighter] (A-2-1.south) -- (C-2-2.south);
\draw[brighter] (B-1-2.west)  -- (C-2-2.west);
\draw[brighter] (B-1-2.east)  -- (C-2-2.east);

\draw[<->,darker](A-2-1) to[in=180,out=90]
    node[arrow style mul] (x) {\(a_{21}\times b_{12}\)} (B-1-2);
\draw[<->,darker](A-2-2) to[in=180,out=90]
    node[arrow style mul] (y) {\(a_{22}\times b_{22}\)} (B-2-2);
\draw[<->,darker](A-2-4) to[in=180,out=90]
    node[arrow style mul] (z) {\(a_{2p}\times b_{p2}\)} (B-4-2);

\draw[darker,->] (x) to node[arrow style plus] {\(+\)} (y)%
                  to node[arrow style plus] {\(+\raisebox{.5ex}{\ldots}+\)} (z)%
                  to (C-2-2.north west);

\node [draw,below=10pt] at (C.south)
    {\(C = A\times B\) : \textcolor{darker}{\(n\) righe}  \textcolor{darker}{\(q\) colonne}};

\end{tikzpicture}
\end{document}

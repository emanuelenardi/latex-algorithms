\chapter*{Prologo}
\addcontentsline{toc}{chapter}{Prologo}

Caro studente,

Il commento più utili sono quelli che documentano caratteristiche Del codice di non immediata comprensione.
\`{E} ragionevole supporre che chi legge il codice possa puoi capire \emph{cosa} esso faccia; è più utile spiegare perché.

% ho lavorato a questi appunti per un anno cercando di offrirti l'esperienza di lettura e comprensione migliore possibile, rileggendo più volte, correggendo errori e facendole revisionare da alcuni tuoi compagni di corso che mi hanno aiutato a portarti un lavoro corretto e revisionato.
%
% Quindi ti prego di non prendere questo lavoro e fotocopiando gli appunti qui presenti o addirittura rivendendolo.
% Ho davvero sudato per scriverli e sarebbe davvero un peccato se i miei sforzi venissero resi vani in un batter di ciglia.
%
% Ti auguro buon studio e ti ringrazio per avermi incentivato a condividere ulteriormente questo lavoro, migliorato un po' alla volta.
%
% Il costo di stampa e di rilegatura verranno specificati.
%
% Vorrei ringraziare Francesco Bozzo, e Samuele Conti per aver revisionato questa dispensa, per la considerevole pazienza e meticolosità delle correzioni.
% Gli errori rimanenti sono, ovviamente, interamente miei.
% Le carenze di questa dispensa sarebbero considerevolmente maggiori e più numerose se non fosse stato per la loro assistenza.
%
% Ti auguro buono studio e di passare l'esame a pieni voti!

Ho cercato di rispettare la seguente affermazione tratta da Scrivere
\enquote{Il commento più utili sono quelli che documentano caratteristiche del codice di non immediata comprensione.
\`{E} ragionevole supporre che chi legge il codice possa capire cosa esso faccia; è più utile spiegare perché.}

Questa affermazione è sicuramente vera per coloro che hanno già delle forti basi di programmazione, per sfortuna mia e per fortuna per voi, questo non era il mio caso quindi ogni algoritmo è stato analizzato, scomposto e svicerato di ogni sua componente fino ad arrivare al suo principio.

Non ho dato nessun preconcetto per scontato e tutti i passaggi, matematici e logici, sono stati resi esplciciti in ogni momento possibile.

Non soffermarti ad una lettura passiva del testo, dovrebbe essere una traccia delle lezioni, prova invece a provare gli algoritmi prima di entrare in aula e ascoltare una lezione frontale che privo del comprensione a priori di quel che stai vedendo altrimenti per la prima volta può risultare una completa perdita di tempo.

% NOTE prendere spunto da quella guida matematica.

Questo è il mio l\'{a}scito per le generazioni future

Ti auguro buona lettura

Emanuele

Impaginarlo come un libro a capitoli

Prefazione

I seguenti appunti sono una traccia delle lezioni di Algoritmi e Strutture Dati, il mio consiglio per affrontare la materia è quella di leggere prima l'argomento e andare a lezione con le idee ben chiare sull'argomento affrontato a lezione, in questo modo ti verrà spontaneo fare domande su ciò che non ti è chiaro o che eventualmente non hai capito avendo già toccato quell'argomento da solo.
Non aver paura di far domande! Il professore è super disponibile ed è uno dei migliori del nostro ateneo, non si farà problemi a risponderti e anche se la domanda ti sembra banale è molto probabile che tu non sia l'unico ad avere quel dubbio e magari farai un favore a qualcuno che è più timido di te.
E, si sa, interagendo trattai molto più beneficio dalla lezione frontale in quanto porterai a casa un'esperienza individuale e ti sarà più semplice ricordarti quale fosse la tua domanda o il tuo dubbio e la relativa risposta in sede d'esame.

Dall'anno accademico 2018/2019 il corso di Algoritmi è passato dall'essere un corso semestrale ad un corso annuale, questo ha dato a molti studenti la possibilità di avere più tempo per interiorizzare le molte nozioni richieste per una comprensione piena degli argomenti trattati, ma al contempo quando nel mese di marzo riprenderai la materia in mano (che tu abbia affrontato il primo parziale o meno) è il caso che tu riprenda in mano questi argomenti, anche solo per una rilettura veloce.

Ringraziamento

Grazie \nomeAcquirente{}
(\cognomeAcquirente{})

\nomeCompletoAcquirente{}

per acquistato i miei appunti di \subject{}

Ti ringrazio per avermi dato fiducia e per aver apprezzato un lavoro così tanto sudato.
Se trovi qualche errore (sintattici o semantico) non avere remure a segnalarmeli il prima possibile: farai una favore a tutti!
Una versione pubblica e aggiornata dell'errata corrige sarà presente alla pagina (readme su repo GitHub latex-algorithms-errata)

Come ultima battuta vorrei ringraziare tutti coloro che mi hanno sostenuto in questi mesi per la stesura di questo testo e a colore che senza i quali sarebbe rimasti dei semplici appunti:

\chapter*{Prologo}
\addcontentsline{toc}{chapter}{Prologo}

Caro studente,

% ho lavorato a questi appunti per un anno cercando di offrirti l'esperienza di lettura e comprensione migliore possibile, rileggendo più volte, correggendo errori e facendole revisionare da alcuni tuoi compagni di corso che mi hanno aiutato a portarti un lavoro corretto e revisionato.
%
% Quindi ti prego di non prendere questo lavoro e fotocopiando gli appunti qui presenti o addirittura rivendendolo.
% Ho davvero sudato per scriverli e sarebbe davvero un peccato se i miei sforzi venissero resi vani in un batter di ciglia.
%
% Ti auguro buon studio e ti ringrazio per avermi incentivato a condividere ulteriormente questo lavoro, migliorato un po' alla volta.
%
% Il costo di stampa e di rilegatura verranno specificati.
%
% Vorrei ringraziare Francesco Bozzo, e Samuele Conti per aver revisionato questa dispensa, per la considerevole pazienza e meticolosità delle correzioni.
% Gli errori rimanenti sono, ovviamente, interamente miei.
% Le carenze di questa dispensa sarebbero considerevolmente maggiori e più numerose se non fosse stato per la loro assistenza.
%
% Ti auguro buono studio e di passare l'esame a pieni voti!

Ho cercato di rispettare la seguente affermazione tratta da Scrivere
\enquote{Il commento più utili sono quelli che documentano caratteristiche del codice di non immediata comprensione.
\`{E} ragionevole supporre che chi legge il codice possa capire cosa esso faccia; è più utile spiegare perché.}

Questa affermazione è sicuramente vera per coloro che hanno già delle forti basi di programmazione, per sfortuna mia e per fortuna per voi, questo non era il mio caso quindi ogni algoritmo è stato analizzato, scomposto e svicerato di ogni sua componente fino ad arrivare al suo principio.

Non ho dato nessun preconcetto per scontato e tutti i passaggi, matematici e logici, sono stati resi esplciciti in ogni momento possibile.

Non soffermarti ad una lettura passiva del testo, dovrebbe essere una traccia delle lezioni, prova invece a provare gli algoritmi prima di entrare in aula e ascoltare una lezione frontale che privo del comprensione a priori di quel che stai vedendo altrimenti per la prima volta può risultare una completa perdita di tempo.

% NOTE prendere spunto da quella guida matematica.

Questo è il mio l\'{a}scito per le generazioni future

Ti auguro buona lettura

Emanuele

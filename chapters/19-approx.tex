%&../settings/preamble.main

% NOTE per controllare dove sfori la pagina
\setlength\overfullrule{5pt}

\ifloadsubpreamble
\pagestyle{plain}
\setcounter{chapter}{18}
\fi

% arara: pdflatex: { draft: yes, synctex: no }
% arara: pdflatex: { synctex: no }
% arara: latexmk:  { clean: partial }
\begin{document}
\tableofcontents

\chapter{Soluzioni per problemi intrattabili}

\section{Algoritmi pseudo-polinomiali}

Non si può avere tutto dalla vita; bisogna rinunciare a qualcosa:
\begin{itemize}
	\item \textbf{generalità}: algoritmi pseudo-polinomiali che funzionano per solo alcuni casi particolari dell'input;
	\item \textbf{ottimalità}: algoritmi di approssimazione, che garantiscono di ottenere soluzioni \enquote{vicine} alla soluzione ottimale;
	\item \textbf{formalità}: algoritmi euristici, di solito basati su tecniche \foreign{greedy} o di ricerca locale, che forniscano sperimentalmente risultati buoni;
	\item \textbf{efficienza}: algoritmi esponenziali branch-\&-bound, che limitano lo spazio di ricerca con un'accurata potatura.
\end{itemize}

\subsection{Somma di sottoinsieme}

\paragraph{Definizione del problema}
Dati un insieme \(A = \{a_1, a_2, \dots, a_n\}\) di interi positivi e un intero positivo \(k\), esiste un sottoinsieme \(S\) di indici in \(\{1, \dots, n\}\) tale che \(\sum_{i \in S}^{a_i} = k\) ?

Utilizzando \foreign{backtracking}, abbiamo risolto la versione di ricerca di questo problema.
Quella appena enunciata è la versione decisionale.
Per semplificare il confronto, ci concentriamo sulla seconda.

\subsubsection{Somma di sottoinsiemi risolto tramite programmazione dinamica}

Definiamo una tabella booleana \(DP[0 \dots n][0 \dots k]\).
\(DP[i][r]\) è uguale a \True se e solo se è possibile ottenere \(r\) dai primi \(i\) valori memorizzati nel vettore di input.
\[
	DP[i][r] =
	\begin{dcases}
		\True                                         & r = 0 \\
		\False                                        & r > 0 \land i = 0\\
		DP[i-1][r]                                    & r > 0 \land i > 0 \land A[i] > r\\
		DP[i - 1][r] \text{ \Or } DP[i - 1][r - A[i]] & r > 0 \land i > 0 \land A[i] \leqslant r\\
	\end{dcases}
\]

Essendo un problema decisionale, è possibile semplificare e utilizzare spazio \(\Theta(k)\), invece che \(\Theta(nk)\), in quanto avrò \(n\) righe e \(k\) colonne.
L'algoritmo sfrutta l'equazione di ricorrenza.

\begin{algorithm}[H]
	\caption{Somma di sottoinsiemi risolto tramite programmazione dinamica}
	%&../preamble

% arara: pdflatex: { synctex: no }
% arara: latexmk: { clean: partial }
\ifstandalone
\begin{document}
\begin{algorithm}[H]
\fi

\prototype{\Bool \subSetSum{\Array{\Int} \(A\), \Int \(n\), \Int \(k\)}}{

	\BlankLine
	\Matrix{\Bool} \(DP\) \Assign \new \Matrix{\Bool}[0\dots n][0\dots k] = \{{\False}\}\;

	\BlankLine
	\tcp{CASI BASE}

	\BlankLine
	\tcp{se il valore da ottenere è 0, non ho bisogno di selezionare nessun indice}
	\For(\tcp*[h]{\(prima colonna\)}){\(i \Assign 0\) \DownTo \(n\)}{

		\BlankLine
		\(DP[i][0] \Assign \True\) \tcp{\(r=0\)}
	}

	\BlankLine
	\tcp{se non ho nessun intero positivo da selezionare non mi è possibile arrivare al valore richiesto}
	\For(\tcp*[h]{\(prima riga\)}){\(r \Assign 1\) \DownTo \(k\)}{

		\BlankLine
		\(DP[0][r] \Assign \False\) \tcp{\(r>0 \land i=0\)}
	}

	\BlankLine
	\tcp{CASO RICORSIVO}
	\From{\(i \Assign 1\) \DownTo \(n\)}{
		\From(\tcp*[h]{\(commento\)}){\(r \Assign 1\) \DownTo \(A[i]-1\)}{

			\BlankLine
			\(DP[i][r] \Assign DP[i-1][r]\) \tcp{\(A[i] > r\)}
		}

		\BlankLine
		\From(\tcp*[h]{\(commento\)}){\(r \Assign A[i]\) \DownTo \(k\)}{

			\BlankLine
			\(DP[i][r] \Assign DP[i-1][r]\) \Or \(DP[i-1][r - A[i]]\) \tcp{\(A[i] \leqslant r\)}
		}
	}

	\BlankLine
	\tcp{restituisco il valore in posizione \(k\)-esima}
	\Return \(DP[n][k]\)\;
}

\ifstandalone
\end{algorithm}
\end{document}
\fi

\end{algorithm}

Ad esempio prendendo l'insieme \(A = [5,9,10]\) di interi positivi e l'intero positivo \(k = 24\) proviamo a riempire la tabella di programmazione dinamica.
Dove le righe rappresentano gli indici presi e le colonne il valore di \(k\) desiderato.

\begin{table}[H]\centering
	\begin{tabular}{@{} c *{25}{@{\hskip 8pt}c} @{}}
		\toprule
			& \textbf{0} & \textbf{1} & \textbf{2} & \textbf{3} & \textbf{4} & \textbf{5} & \textbf{6} & \textbf{7} & \textbf{8} & \textbf{9} & \textbf{10} & \textbf{11} & \textbf{12} & \textbf{13} & \textbf{14} & \textbf{15} & \textbf{16} & \textbf{17} & \textbf{18} & \textbf{19} & \textbf{20} & \textbf{21} & \textbf{22} & \textbf{23} & \textbf{24}\\
		\cmidrule{2-26}
			\textbf{0} & 1 &&&&&&&&&&&&&&&&&&&&&&&& \\
		\lightrule
			\textbf{1} & 1 &&&&& 1 &&&&&&&&&&&&&&&&&&& \\
		\lightrule
			\textbf{2} & 1 &&&&& 1 &&&& 1 &&&&& 1 &&&&&&&&&& \\
		\lightrule
			\textbf{3} & 1 &&&&& 1 &&&& 1 & 1 &&&& 1 & 1 &&&& 1 &&&&& 1 \\
		\bottomrule
	\end{tabular}
\end{table}

\paragraph{Analisi della complessità}
La complessità dell'algoritmo è \(\Theta(nk)\), ma la complessità dei dati in ingresso è \(\Omicron(n \log k)\), in quanto i valori più grandi del nostro obiettivo possono essere esclusi.
Se \(k\) è \(\Omicron(n^c)\) con \(c\) costante, allora \subSetSum ha complessità polinomiale \(\Omicron(n^{c+1})\).
Ma se \(k\) è  \(\Omicron(2^n)\), allora \subSetSum ha complessità superpolinomiale \(\Omicron(n \cdot 2^{n})\).

\begin{observation}
La complessità di \subSetSum dipende quindi dai valori contenuti nell'insieme e non soltanto dalla cardinalità dei dati in ingresso (\(n\)).
\end{observation}

\subsubsection{Somma di sottoinsiemi risolto tramite backtracking}

\begin{algorithm}[H]
	\caption{Somma di sottoinsiemi risolto tramite backtracking}
	%&../preamble

% arara: pdflatex: { synctex: no }
% arara: latexmk: { clean: partial }
\ifstandalone
\begin{document}
\begin{algorithm}[H]
\fi

\prototype{\Bool \subSetSumRec{\Array{Int} \(A\), \Int \(i\), \Int \(r\)}}{

	\BlankLine
	\uIf{\(r \Equal 0\)}{
		\Return \True\;
		\BlankLine
	}
	\uElseIf{\(i \Equal 0\)}{
		\Return \False\;
		\BlankLine
	}
	\uElseIf{\(A[i] > r\)}{
		\Return \(\subSetSumRec(A, i-1, r)\)\;
		\BlankLine
	}
	\Else{
		\Return \(\subSetSumRec(A, i-1, r)\) \Or \(\subSetSumRec(A, i-1, r-A[i])\)\;
	}

	\BlankLine
	\Return \(res\)\;
}

\ifstandalone
\end{algorithm}
\end{document}
\fi

\end{algorithm}

\begin{table}[H]\centering
	\begin{tabular}{@{} c *{25}{@{\hskip 8pt}c} @{}}
		\toprule
			& \textbf{0} & \textbf{1} & \textbf{2} & \textbf{3} & \textbf{4} & \textbf{5} & \textbf{6} & \textbf{7} & \textbf{8} & \textbf{9} & \textbf{10} & \textbf{11} & \textbf{12} & \textbf{13} & \textbf{14} & \textbf{15} & \textbf{16} & \textbf{17} & \textbf{18} & \textbf{19} & \textbf{20} & \textbf{21} & \textbf{22} & \textbf{23} & \textbf{24}\\
		\cmidrule{2-26}
			\textbf{0} & 1 &&&&& 0 &&&& 0 & 0 &&&& 0 & 0 &&&&&&&&& 0 \\
		\lightrule
			\textbf{1} &&&&&& 1 &&&&&&&&& 0 & 0 &&&&&&&&& 0 \\
		\lightrule
			\textbf{2} &&&&&&&&&&&&&&& 1 &&&&&&&&&& 0 \\
		\lightrule
			\textbf{3} &&&&&&&&&&&&&&&&&&&&&&&&& 1 \\
		\bottomrule
	\end{tabular}
\end{table}

\paragraph{Complessità}
\(\Omicron(2^n)\)

\subsubsection{Somma di sottoinsiemi risolto tramite memoization}

\begin{algorithm}[H]
	\caption{Somma di sottoinsiemi risolto tramite memoization}
	%&../preamble

% arara: pdflatex: { synctex: no }
% arara: latexmk: { clean: partial }
\ifstandalone
\begin{document}
\begin{algorithm}[H]
\fi

\prototype{\Bool \subSetSumMemo{\Array{\Int} \(A\), \Int \(i\), \Int \(r\), \dictionary \(DP\)}}{

	\BlankLine
	\uIf{\(r \Equal 0\)}{
		\Return \True\;
		\BlankLine
	}
	\uElseIf{\(r < 0\) \Or \(i \Equal 0\)}{
		\Return \False\;
		\BlankLine
	}
	\Else{
		\Bool \(res\) \Assign \(DP.\dictLookup((i,r))\)\;

		\BlankLine
		\If{\(res \Equal \Nil\)}{

			\BlankLine
			\If{\(A[i] < r\)}{
				\(res\) \Assign \(res\) \Or \(\subSetSumMemo(A, i-1, r-A[i], DP)\)\;
			}

			\BlankLine
			\(DP.\dictInsert((i,1), res)\)\;
		}

		\BlankLine
		\Return \(res\)\;
	}
}

\ifstandalone
\end{algorithm}
\end{document}
\fi

\end{algorithm}

\begin{table}[H]\centering
	\begin{tabular}{@{} c *{25}{@{\hskip 8pt}c} @{}}
		\toprule
			& \textbf{0} & \textbf{1} & \textbf{2} & \textbf{3} & \textbf{4} & \textbf{5} & \textbf{6} & \textbf{7} & \textbf{8} & \textbf{9} & \textbf{10} & \textbf{11} & \textbf{12} & \textbf{13} & \textbf{14} & \textbf{15} & \textbf{16} & \textbf{17} & \textbf{18} & \textbf{19} & \textbf{20} & \textbf{21} & \textbf{22} & \textbf{23} & \textbf{24}\\
		\cmidrule{2-26}
			\textbf{0} & 1 &&&&& 0 &&&& 0 & 0 &&&& 0 & 0 &&&&&&&&& 0 \\
		\lightrule
			\textbf{1} &&&&&& 1 &&&&&&&&& 0 & 0 &&&&&&&&& 0 \\
		\lightrule
			\textbf{2} &&&&&&&&&&&&&&& 1 &&&&&&&&&& 0 \\
		\lightrule
			\textbf{3} &&&&&&&&&&&&&&&&&&&&&&&&& 1 \\
		\bottomrule
	\end{tabular}
\end{table}

Se prendiamo in considerazione un altro esempio con l'insieme \(A = [1,1,1,1,1]\) di interi positivi e l'intero positivo \(k = 5\) proviamo a riempire la tabella di programmazione dinamica

\begin{table}[H]\centering
	\begin{tabular}{@{} c *{6}{@{\hskip 16pt}c} @{}}
		\toprule
			& \textbf{0} & \textbf{1} & \textbf{2} & \textbf{3} & \textbf{4} & \textbf{5} \\
		\cmidrule{2-7}
			\textbf{0} & 1 & 0 & 0 & 0 & 0 & 0 \\
		\lightrule
			\textbf{1} &   & 1 & 0 & 0 & 0 & 0 \\
		\lightrule
			\textbf{2} &   &   & 1 & 0 & 0 & 0 \\
		\lightrule
			\textbf{3} &   &   &   & 1 & 0 & 0 \\
		\lightrule
			\textbf{4} &   &   &   &   & 1 & 0 \\
		\lightrule
			\textbf{5} &   &   &   &   &   & 1 \\
		\bottomrule
	\end{tabular}
\end{table}

\paragraph{Complessità}
\(\Omicron(nk)\)

\subsubsection*{Discussione sulla complessità di somma di sottoinsiemi}

Riassumendo abbiamo risolto il problema tramite la programmazione dinamica ottenendo una complessità di \(\Theta(nk)\), tramite backtracking ottenendo una complessità di \(\Omicron(2^n)\) ed infine tramite memoization ottenendo una complessità di \(\Omicron(nk)\).

Ma \(\Omicron(nk)\) è una complessità superpolinomiale?
No, non lo è, infatti \(k\) è parte dell'input, non una dimensione dell'input.
\(k\) viene rappresentato da \(t = \ceil{\log k}\) cifre binarie.
Quindi la complessità è \(\Omicron(nk) = \Omicron(n \dots 2^t)\), esponenziale.

\section{Problemi fortemente, debolmente {\NP}-completi}

\begin{definition}[dimensioni del problema]
Dato un problema decisionale \(R\) e una sua istanza \(I\).
La \alert{dimensione \(d\)} di \(I\) è la lunghezza della stringa che codifica \(I\).
Il \alert{valore \texttt{\#}} è il più grande numero intero che appare in \(I\).
\end{definition}

\begin{table}[htbp]\centering
	\caption{Problemi decisionali e relative grandezze}
	\label{tab:problem-dimensions}
	\begin{tabular}{@{} l *{3}{l} @{}}
		\toprule
			\textbf{Nome} & \multicolumn{1}{c}{\textbf{Istanza} \(I\)} & \multicolumn{1}{c}{\textbf{No.\ più grande} \texttt{\#}} & \multicolumn{1}{c}{\textbf{Dimensione} \(d\)} \\
		\midrule
			{\subSetSumProblem} & \(\{n, k, A\}\)    & \(\max\{n, k, \max(A)\)\} & \(\Omicron(n \log\texttt{\#})\)\\
		\lightrule
			{\cliqueProblem}    & \(\{n, m, k, G\}\) & \(\max\{n,m,k\}\) & \(\Omicron(n + m + \log\texttt{\#})\)\\
		\lightrule
			{\tsp} & \(\{n,k,d\}\) & \(\max\{n, k, \max(d)\}\) & \(\Omicron(n^2 \log\texttt{\#})\)\\
		\bottomrule
	\end{tabular}
\end{table}

\begin{definition}[problema fortemente \NP-completo]
Sia \(R_p\) il problema \(R\) ristretto a quei dati d'ingresso per i quali il più grande valore da rappresentare è limitato superiormente da \(p(d)\), con \(p\) funzione polinomiale in \(d\).
\(R\) è \alert{fortemente \NP-completo} se \(R_p\) è \NP-completo.
\end{definition}

\begin{observation}
Il problema cricca ({\cliqueProblem}), ad esempio, è fortemente \NP-completo, mentre quello della somma di sottoinsieme {\subSetSumProblem} non lo è.
\end{observation}

\begin{definition}[problema debolmente \NP-completo]
Se un problema \NP-completo non è fortemente \NP-completo, allora è \alert{debolmente \NP-completo}.
\end{definition}

\subsection{Esempio di problema debolmente \NP-completo}

\subsubsection{Somma di sottoinsiemi (\subSetSumProblem)}

\paragraph{Definizione del problema}
Dati un vettore \(A\) contenente \(n\) interi positivi ed un intero positivo \(k\), \alert{esiste} un sottoinsieme \(S \subseteq \{1 \dots n\}\) tale che \(\sum_{i \in S} a[i] = k\)?

\begin{proof}[Dimostrazione. Somma di sottoinsieme è debolmente \NP-completo]
\(\forall A[i] \leqslant k\) (valori più grandi di \(k\) vanno esclusi).
Se \(k = \Omicron(n^c)\), allora \(\texttt{\#} = \max\{n, k, a_1, \dots, a_n\} = \Omicron(n^c)\).
La soluzione basata su programmazione dinamica ha complessità \(\Omicron(nk) = \Omicron(n^{c+1})\), quindi in {\PTIME}.
Possiamo dedurne che {\subSetSumProblem} non è fortemente \NP-completo.
\end{proof}

\begin{note}
Il problema di {\subSetSumProblem} è debolmente \NP-completo, in quanto limitando le dimensioni dei dati in ingresso il problema diventa polinomiale.
\end{note}

\subsection{Algoritmi pseudo-polinomiali}

\begin{definition*}[Algoritmo pseudo-polinomiale]
Un algoritmo che risolve un certo problema \(R\), per qualsiasi dato \(I\) d'ingresso, in tempo \(p(\texttt{\#},d)\), con \(p\) funzione polinomiale in \texttt{\#} e \(d\), ha complessità \alert{pseudo-polinomiale}.
\end{definition*}

\begin{observation}
Gli algoritmi per {\subSetSumProblem} basati su programmazione dinamica e memoization sono pseudo-polinomiali.
\end{observation}

\begin{theorem*}
Nessun problema fortemente \NP-completo può essere risolto da un algoritmo pseudo-polinomiale, a meno che non sia \(\PTIME = \NP\).
\end{theorem*}

\subsection{Esempi di problemi fortemente \NP-completo}

\subsubsection{Cricca (\cliqueProblem)}

\paragraph{Definizione del problema}
Dati un grafo non orientato ed un intero \(k\), esiste un sottoinsieme di almeno \(k\) nodi tutti mutuamente adiacenti?

\begin{proof}[Dimostrazione. {\cliqueProblem} è fortemente \NP-completo]
\(k \leqslant n\) (altrimenti la risposta è \False), \(\texttt{\#} = \max\{n,m,k\}\), in quanto \(k \leqslant n\) possiamo semplificare \(\texttt{\#} = \max\{n,m\}\), \(d = \Omicron(n + m + \log\texttt{\#}) = \Omicron(n + m)\).
Quindi \(\texttt{\#} = \max\{n, m\}\) è limitato superiormente da \(\Omicron(n + m)\).
Possiamo dedurne che il problema ristretto è identico a {\cliqueProblem}, che è \NP-completo.
\end{proof}

\subsubsection{Commesso viaggiatore}

\paragraph{Definizione del problema}
Date \(n\) città e una matrice simmetrica d di distanze positive, dove \(d[i][j]\) è la distanza fra \(i\) e \(j\), trovare un percorso che, partendo da una qualsiasi città, attraversi ogni città esattamente una volta e ritorni alla città di partenza, in modo che la distanza totale percorsa sia minima.

\begin{proof}[Dimostrazione per assurdo che {\tsp} è fortemente \NP-completo]
Per assurdo, supponiamo che {\tsp} sia debolmente \NP-completo.
Allora esiste una soluzione pseudo-polinomiale.
Usiamo questa soluzione per risolvere un problema \NP-completo in tempo polinomiale, il che è assurdo a meno che \(\PTIME = \NP\).
\end{proof}

\subsubsection{Circuito hamiltoniano}

\paragraph{Definizione del problema}
Dato un grafo non orientato \(G\), esiste un circuito che attraversi ogni nodo una e una sola volta?

\paragraph{Considerazioni}
Il problema {\hamiltonianCircuit} è \NP-completo.
\'{E} uno dei 21 problemi elencati nell'articolo di Karp.

\begin{proof}[Dimostriamo che {\tsp} è fortemente \NP-completo.]
Sia \(G = (V,E)\) un grafo non orientato.
Definiamo una matrice di distanze a partire da \(G\).
\[
	d[i][j] =
	\begin{dcases}
		1 & (i,j) \in E\\
		2 & (i,j) \not\in E\\
	\end{dcases}
\]

Il grafo \(G\) ha un circuito hamiltoniano se e solo se è possibile trovare un percorso da commesso viaggiatore di costo \(n\).

\paragraph{Simmetria fra {\tsp} e {\hamiltonianCircuit}}
Se esistesse un algoritmo pseudopolinomiale \(A\) per {\tsp}, {\hamiltonianCircuit} potrebbe essere risolto da \(A\) in tempo polinomiale.
\end{proof}

\subsubsection{Partizione (\partition)}

\paragraph{Definizione del problema}
Dato un vettore \(A\) contenente \(n\) interi positivi, esiste un sottoinsieme \(S \subseteq \{1 \dots n\}\) tale che \(\sum_{i \in S} A[i] = \sum_{i \in S} A[i]\)?

\begin{figure}[H]
	\centering
	\begin{tikzpicture}
		\coordinate (t) at (.25,+.6);
		\foreach \num in {0,1,...,5}{
			\node[font=\tiny] at (t) {\num};
			\coordinate (t) at ($(t) + (0.6,0)$);
		}

		\coordinate (s) at (0,0);
		\foreach \num in {14,6,12,3,7,2}{
			% \node[draw, thick, rectangle, minimum size=0.5cm] at (s) {\num};
			\node[cell, minimum width = 0.6cm, minimum height = 0.6cm] at (s) {\num};
			\coordinate (s) at ($(s) + (0.6,0)$);
		}
	\end{tikzpicture}
\end{figure}

\paragraph{Conclusioni}
Il problema {\partition} è debolmente \NP-completo perché è possibile ridurlo a {\subSetSumProblem} scegliendo come valore \(k\) la metà di tutti i valori presenti:
\[
	k = \frac{\sum_{i=1}^{n} A[i]}{2} = \frac{44}{2} = 22
\]

\subsubsection{3-Partizione (\treePartition)}

Dati \(3n\) interi \(\{a_1, \dots, a_{3n}\}\) esiste una partizione in \(n\) triple \(T_1, \dots, T_n\), tale che la somma dei tre elementi di ogni \(T_j\) sia la stessa, per \(1 \leqslant j \leqslant n\)?

\paragraph{Conclusioni}
Il problema {\treePartition} è fortemente \NP-completo perché non esiste un algoritmo pseudopolinomiale per risolverlo.

\section{Algoritmi di approssimazione}

Facciamo una piccola premessa.
I problemi più interessanti sono in forma di ottimizzazione.
Se il problema di decisione è \NP-completo, non sono noti algoritmi polinomiali per il problema di ottimizzazione.
Esistono algoritmi polinomiali che trovano soluzioni ammissibili più o meno vicine a quella ottima.

\begin{definition*}[Algoritmi di approssimazione]
Se è possibile dimostrare un limite superiore/inferiore al rapporto fra la soluzione trovata e la soluzione ottima, allora tali algoritmi vengono detti \alert{algoritmi di approssimazione}.
\end{definition*}

\begin{definition*}[\(\alpha(n)-approssimazione\)]
Dato un problema di ottimizzazione con funzione costo non negativa \(c\), un algoritmo si dice di \alert{\(\alpha(n)\)-approssimazione} se fornisce una soluzione ammissibile \(x\) il cui costo \(c(x)\) non si discosta dal costo \(c(x^*)\) della soluzione ottima \(x^*\) per più di un fattore \(\alpha(n)\), per qualunque input di dimensione \(n\):
\begin{alignat}{4}
	c(x^*) \leqslant c(x)           &\leqslant \alpha(n) c(x^*) &\quad& \alpha(n) > 1 \quad (\text{Minimizzazione})\\
	\alpha(n) c(x^*) \leqslant c(x) &\leqslant c(x^*)           &\quad& \alpha(n) < 1 \quad (\text{Massimizzazione})
\end{alignat}
\end{definition*}

\begin{observation}
\(\alpha(n)\) può essere una costante, valida per tutti gli \(n\).
\end{observation}

\begin{note}
Identificare un valore \(\alpha(n)\) e dimostrare che l'algoritmo lo rispetta è ciò che rende un buon algoritmo un algoritmo di approssimazione.
\end{note}

\subsection{Bin packing}

\paragraph{Definizione del problema}
Dati un vettore \(A\) contenente \(n\) interi positivi (i \alert{volumi} degli \alert{oggetti}) e un intero positivo \(k\) (la \alert{capacità} di una \alert{scatola}, tale che \(\forall i: A[i] \leqslant k\)), si vuole trovare una partizione di \(\{1, \dots, n\}\) nel minimo numero di sottoinsiemi disgiunti (\enquote{scatole}) tali che \(\sum_{i \in S} A[i] \leqslant k\) per ogni insieme \(S\) della partizione.

Dato il seguente vettore e un intero \(k = 8\) come risolvereste il problema?
\begin{figure}[H]
	\centering
	\begin{tikzpicture}
		\coordinate (t) at (.25,+.5);
		\foreach \num in {0,1,...,6}{
			\node[font=\tiny] at (t) {\num};
			\coordinate (t) at ($(t) + (0.5,0)$);
		}

		\coordinate (s) at (0,0);
		\foreach \num in {3,7,2,5,4,3,5}{
			\node[cell] at (s) {\num};
			\coordinate (s) at ($(s) + (0.5,0)$);
		}
	\end{tikzpicture}
\end{figure}

\subsubsection{Appoccio ingordo per il problema bin packing}

\paragraph{Algoritmo first-fit}
Gli oggetti sono considerati in un ordine qualsiasi e ciascun oggetto è assegnato alla prima scatola che lo può contenere, tenuto conto di quanto spazio è stato occupato della stessa.

\begin{figure}[H]
	\centering
	\begin{tikzpicture}

		\coordinate (s) at (0,0);
		\foreach \num in {{3,2,3},7,5,4,5}{
			\node[cell, minimum width = 1.3cm] at (s) {\num};
			\coordinate (s) at ($(s) + (1.3,0)$);
		}
	\end{tikzpicture}
\end{figure}

Sia \(N > 1\) il numero di scatole usare da {\firstFit} (se \(N=1\), l'algoritmo è ottimale).
Il numero minimo di scatole \(N^*\) è limitato da:
\[
	N^{*} \geqslant \frac{\sum_{i=1}^n A[i]}{k} = \frac{29}{8} = 3.625
\]
non possono esserci due scatole riempite meno della metà:
\[
	N < \frac{\sum_{i=1}^n A[i]}{k/2} = \frac{29}{8/2} = 7.250
\]
abbiamo quindi:
\[
	N < \frac{\sum_{i=1}^n A[i]}{k/2} = 2 \frac{\sum_{i=1}^n A[i]}{k} \leqslant 2N^* = \alpha(n) N^{*}
\]
che implica \(\alpha(n) = 2\).

\paragraph{Algoritmo first-fit decreasing}
Se consideriamo gli oggetti in ordine non descrescente è possibile dimostrare un risultato migliore per questo algoritmo.
\[
	N < \frac{17}{10} N^{*} + 2
\]

Nella variante \textsc{ffd} (First-fit decreasing): gli oggetti sono considerati in ordine non decrescente
\[
	N < \frac{11}{9} N^{*} + 4
\]
Queste sono dimostrazioni di limiti superiori per il fattore \(\alpha(n)\), per casi particolari l'approssimazione può essere migliore.

\subsection[Commesso viaggiatore con disuguaglianze triangolari]
		   {Commesso viaggiatore con disuguaglianze triangolari ({\deltaTsp})}

\paragraph{Definizione del problema}
Siano date \(n\) città e le distanze (positive) \(d[i][j]\) tra esse, \alert{tale per cui vale la regola delle diseguaglianze triangolari}:
\[
	d[i][j] \leqslant d[i][k] + d[k][j] \quad \forall i,j,k: \quad 1 \leqslant i,j,k \leqslant n
\]
Trovare un percorso che, partendo da una qualsiasi città, attraversi ogni città esattamente una volte e ritorni alla città di partenza, in modo che la distanza complessiva percorsa sia minima.

\begin{figure}[H]
	\begin{subfigure}[t]{.5\linewidth}\centering
		\includegraphics[page=1]{triangle-inequality}
		\caption{con diseguaglianza triangolare}
	\end{subfigure}%
	\begin{subfigure}[t]{.5\linewidth}\centering
		\includegraphics[page=2]{triangle-inequality}
		\caption{senza diseguaglianza triangolare}
	\end{subfigure}
\end{figure}

\begin{proof}[Dimostriamo che {\hamiltonianCircuit} \(\leqslant_p\) {\deltaTsp}]
Sia \(G = (V, E)\) un grafo non orientato. Definiamo una matrice delle distanze a partire da \(G\)
\[
	d[i][j] =
	\begin{dcases}
		1 & (i,j) \in E\\
		2 & (i,j) \not\in E\\
	\end{dcases}
\]
Il grafo \(G\) ha un circuito hamiltoniano se e solo se è possibile trovare un percorso da commesso viaggiatore lungo \(n\).
Valgono le diseguaglianze triangolari:
\[
	d[i][j] \leqslant 2 \leqslant d[i][k] + d[k][j]
\]
\end{proof}

\subsubsection{Commesso viaggiatore come circuito hamiltoniano pesato}

Interpretiamo {\deltaTsp} come il problema di trovare un circuito hamiltoniano di peso minimo su un grafo completo.

% NOTE includere grafica
% \begin{figure}[H]\centering
% 	\includegraphics{tsp-greedy}
% \end{figure}

\begin{figure}[H]\centering
	\begin{subfigure}[t]{.25\linewidth}\centering
		\includegraphics[page=2, width=\linewidth]{tsp-pentagon}
		\caption{}
	\end{subfigure}%
	\begin{subfigure}[t]{.25\linewidth}\centering
		\includegraphics[page=3, width=\linewidth]{tsp-pentagon}
		\caption{}
	\end{subfigure}%
	\begin{subfigure}[t]{.25\linewidth}\centering
		\includegraphics[page=4, width=\linewidth]{tsp-pentagon}
		\caption{}
	\end{subfigure}%
	\begin{subfigure}[t]{.25\linewidth}\centering
		\includegraphics[page=5, width=\linewidth]{tsp-pentagon}
		\caption{}
	\end{subfigure}%

	\vspace{10pt}

	\begin{subfigure}[t]{.25\linewidth}\centering
		\includegraphics[page=6, width=\linewidth]{tsp-pentagon}
		\caption{}
	\end{subfigure}%
	\begin{subfigure}[t]{.25\linewidth}\centering
		\includegraphics[page=7, width=\linewidth]{tsp-pentagon}
		\caption{}
	\end{subfigure}%
	\begin{subfigure}[t]{.25\linewidth}\centering
		\includegraphics[page=8, width=\linewidth]{tsp-pentagon}
		\caption{}
	\end{subfigure}%
	\begin{subfigure}[t]{.25\linewidth}\centering
		\includegraphics[page=9, width=\linewidth]{tsp-pentagon}
		\caption{}
	\end{subfigure}%
	\caption{Commesso viaggiatore pentagon}
\end{figure}

Il costo totale risulta 21.

\begin{figure}[H]\centering
	\begin{subfigure}[t]{.20\linewidth}\centering
		\includegraphics[page=1, width=\linewidth]{tsp-local}
		\caption{}
	\end{subfigure}%
	\begin{subfigure}[t]{.20\linewidth}\centering
		\includegraphics[page=2, width=\linewidth]{tsp-local}
		\caption{}
	\end{subfigure}%
	\begin{subfigure}[t]{.20\linewidth}\centering
		\includegraphics[page=3, width=\linewidth]{tsp-local}
		\caption{}
	\end{subfigure}%
	\begin{subfigure}[t]{.20\linewidth}\centering
		\includegraphics[page=4, width=\linewidth]{tsp-local}
		\caption{}
	\end{subfigure}%
	\hfill
	\begin{subfigure}[t]{.20\linewidth}\centering
		\includegraphics[page=5, width=\linewidth]{tsp-local}
		\caption{}
	\end{subfigure}%
	\caption{Commesso viaggiatore locale}
\end{figure}

Il costo totale risulta 19.

\subsubsection{Algoritmo di approssimazione per {\deltaTsp}}

Se si considera un circuito hamiltoniano e si cancella un suo arco, si ottiene un albero di copertura.

% \begin{figure}[H]\centering
% 	\includegraphics[width=1\textwidth]{toscana}
% 	\caption{didascalia}%
% 	\label{fig:etichetta}
% \end{figure}

\begin{theorem*}
Qualunque circuito hamiltoniano \(\pi\) ha costo \(c(\pi)\) superiore al costo \(mst\) di albero di copertura di peso minimo, ovvero \(mst < c(\pi)\).
\end{theorem*}

\begin{proof}[Dimostrazione per assurdo.]
Supponiamo che esista un circuito hamiltoniano \(\pi\) di costo \(c(\pi) \leqslant mst\).
Togliamo un arco, otteniamo un albero di copertura con peso inferiore \(mst' < c(\pi) \leqslant mst\).
Contraddizione, visto che \(mst\) è il costo minimo fra tutti gli alberi di copertura.
\end{proof}

\subsubsection{Algoritmo per {\deltaTsp}}

Si individua un minimo albero di copertura di peso \(mst\) e se ne percorrono gli archi due volte, prima in un senso e poi nell'altro.
In questo modo, si visita ogni città almeno una volta.
La distanza complessiva di tale circuito è uguale a \(2 \cdot mst\).
Ma non è un circuito hamiltoniano!

\begin{figure}[H]\centering
	\begin{subfigure}[t]{.5\linewidth}\centering
		\includegraphics[page=1]{toscana-mst}
		\caption{con diseguaglianza triangolare}
	\end{subfigure}%
	\begin{subfigure}[t]{.5\linewidth}\centering
		\includegraphics[page=2]{toscana-mst}
		\caption{senza diseguaglianza triangolare}
	\end{subfigure}
\end{figure}



% \begin{figure}[H]\centering
% 	\begin{subfigure}[t]{.20\linewidth}\centering
% 		\includegraphics[page=1, width=\linewidth]{tsp-nn}
% 		\caption{}
% 	\end{subfigure}%
% 	\begin{subfigure}[t]{.20\linewidth}\centering
% 		\includegraphics[page=2, width=\linewidth]{tsp-nn}
% 		\caption{}
% 	\end{subfigure}%
% 	\begin{subfigure}[t]{.20\linewidth}\centering
% 		\includegraphics[page=3, width=\linewidth]{tsp-nn}
% 		\caption{}
% 	\end{subfigure}%
% 	\begin{subfigure}[t]{.20\linewidth}\centering
% 		\includegraphics[page=4, width=\linewidth]{tsp-nn}
% 		\caption{}
% 	\end{subfigure}%
% 	\begin{subfigure}[t]{.20\linewidth}\centering
% 		\includegraphics[page=5, width=\linewidth]{tsp-nn}
% 		\caption{}
% 	\end{subfigure}%
% 	\caption{Commesso viaggiatore nn}
% \end{figure}

% \paragraph{Restringiamo il problema}
% \texttt{MATERIALE MANCANTE}
%
% \begin{theorem}[tsp non approssimabile]
% \texttt{MATERIALE MANCANTE}
% \end{theorem}
%
% \begin{note}
% \(\Delta-tsp\) è un problema approssimabile, ma il problema generale no.
% \end{note}
%
% \section{Algoritmi branch\&bound}
%
% Variante della tecnica di programmazione backtrack.
% Cerchiamo di analizare tutto lo spazio delle soluzioni, evitando certe sequence di scelte che facciano diminuire il costo della soluzione parziale costruita.
%
% Cerchiamo i limiti (superiore ed inferiore) della soluzione minima.
%
% \begin{figure}[H]
% 	\centering
% 	\includegraphics{branchAndBound-1}
% 	\caption{Se \textsf{lb}(\(S\),\(i\)) è maggiore o uguale a \(minCost\), allora si può evitare di generare ed esplorare il sottoalbero delle scelte radicato in tal nodo.}
% 	% \label{}
% \end{figure}
%
% Questo metodo non migliora la complessità (superpolinomiale) della procedura \enumeration, ma nella pratica ne abbassa di molto il tempo di esecuzione.
% Tutto dipende dalla funzione \lowerBound, che deve essere il più possibile vicino alla soluzione ottima.
% Il limite superiore è dato dal \(minCost\).
%
% \begin{algorithm}[H]
% 	\caption{Somma di sottoinsiemi}
% 	%&../preamble

% arara: pdflatex: { synctex: no }
% arara: latexmk: { clean: partial }
\ifstandalone
\begin{document}
\begin{algorithm}[H]
\fi

\prototype{\branchAndBound{\datiProblema, \Item{} \(S\), \Int \(i\), \datiParziali}}{
	\Set \(C \Assign \getChoices{\datiProblema, n, i, \datiParziali}\) \Comment*[l]{determina l'insieme in funzione delle scelte precedenti \(S[1 \dots i-1]\)}

	\BlankLine
	\tcp{esamino ogni scelta}
	\ForEach{\(c \in C\)}{

		\BlankLine
		\(S[i] \Assign c\) \tcp{effettuo la scelta \(i\)-esima}

		\BlankLine
		\tcp{calcolo il lower bound in base alle scelte fasse in precedenza}
		\Int \(lb \Assign \lowerBound{\datiProblema, S, i, \datiParziali}\)\;

		\BlankLine
		\If(\tcp*[h]{se il limite inferiore non eccede il costo minimo}){\(lb < minCost\)}{

			\BlankLine
			\uIf(\tcp*[h]{posso ancora effettuare delle scelte}){\(i < n\)}{

				\BlankLine
				\tcp{faccio ricorsivamente le scelte successive}
				\branchAndBound{\datiProblema, S, \(i+1\), \datiParziali}
				\BlankLine

			}
			\Else{
				\tcp{ho effettuato tutte le scelte possibili (\(i \Equal n\))}

				\BlankLine
				\tcp{se il costo della soluzione trovata è minore del costo della soluzione parziale}
				\If{\(\cost(S,i) < minCost\)}{
					\tcp{allora la soluzione trovata è migliore del minimo parziale}

					\BlankLine
					\(minSol \Assign S\) \Comment*[l]{aggiorno la soluzione minima parziale}
					\(minCost \Assign \cost(S,i)\)  \Comment*[l]{aggiorno il costo minimo parziale}
				}
			}
		}

	}
}

\ifstandalone
\end{algorithm}
\end{document}
\fi

% \end{algorithm}
%
% \paragraph{Problema del commesso viaggiatore}
% Applichiamo questo ragionamento al problema del commesso viaggiatore.
% Sia \(n\) il numero delle città, e \(d[h][k]\) la distanza, intera e non negativa, fra le città \(h\) e \(k\).
% Al passo \(i\)-esimo sono state fatte le scelte \(S[1][i]\) prese dall'insieme \(1, \dots, n\).
% Un percorso ammissible che \enquote{espande} \(S[1][i]\) deve
% \begin{enumerate*}[label=\arabic*)]
% 	\item attraversare le città \(S[1][i]\);
% 	\item passare da \(S[i]\) ad una qualsiasi delle rimanenti \(n-1\) città;
% 	\item attraversare queste ultime città in un ordine qualsiasi;
% 	\item da una di queste ritornare a \(S[1]\).
% \end{enumerate*}
%
% Facciamoci un po' di calcoli:
% \(C[i]\) è il costo che ho sostenuto per fare i primi \(i\) passi.
% \begin{equation*}
% 	C[i] =
% 	\begin{dcases}
% 		0 & i = 1 \\
% 		C[i-1] + d[ S[i-1] ][ S[i] ] & i > 1 \\
% 	\end{dcases}
% \end{equation*}
%
% Il limite inferiore (\foreign{lower bound}) della distanza per tornare a \(S[1]\) (\(O(n)\)):
% \[A = \min_{h \notin S} \{ d[S[h][1] \}\]
%
% Il limite inferiore (\foreign{lower bound}) della distanza per andarsene da \(S[i]\) (\(O(n)\)):
% \[B = \min_{h \notin S}\{d[S[i], h] \}\]
%
% Lower bound della distanza percorsa per attraversare una qualsiasi di queste ultime \(n - i\) città, provenendo da (e dirigendosi verso) un'altra di queste \(n - i\) città (\(O(n^3)\))
% \[D[h] = \min_{p,q}\{d[p, h] + d[h, q] \colon h \Neq p \Neq q \},\;\]
%
% Un lower bound \(\lowerBound(S)\) è dato dalla seguente espressione:
% \[
% \lowerBound(s,i) =
% 	\begin{cases}
% 		C[i] + d[S[i], S[1]] & i=n \\
% 		C[i] + A + B + \left\lceil (\sum_{h \notin S} D[h])/2\right\rceil & i < n
% 	\end{cases}
% \]
%
% \begin{algorithm}[H]
% 	\caption{Appoccio branch\&bound al problema del commesso viaggiatore}
% 	\documentclass[varwidth=6in]{standalone}
\usepackage{../_preamble}

% arara: pdflatex: { synctex: no }
% arara: latexmk: { clean: partial }
\begin{document}
\ifstandalone
\NoCaptionOfAlgo
\begin{algorithm}[H]
\caption[Problema del commesso viaggiatore]{}
\fi

\prototype{\bbTsp{\Array{\Item} \(S\), \Array{\Int} \(C\), \Set \(R\), \Int \(n\), \Int \(i\)}}{
	\ForEach{\(c \in R\)}{
		\(S[i] \Assign c\)\;
		\(R.\setRemove{c}\)\;

		\BlankLine
		\(C[i] \Assign C[i-1] + d[S[i-1]][S[i]]\)\;

		\BlankLine
		\{ calcola \(A\), \(B\), \(D[H]\) per ogni \(h \in R\) \}

		\BlankLine
		\Int \(lb \Assign C[i] + \iif{ \(i<n\), \(\Ceil{\frac{\sum_{h \not\in S} D[h]}{2}}\), \(d[S[i]][S[1]]\) }\)\;


		\BlankLine
		\tcp{inizializza \(minCost\) ad una permutazione casuale}
		\(minCost \Assign \random\)

		\BlankLine
		\If{\(lb < minCost\)}{
			\eSea{\(i < n\)}{
				\bbTsp{S, C, R, n, \(i+1\)}\;
			}{
				\(C[n] \Assign lb\)\;
				\(minSol \Assign S\)\;
				\(minCost \Assign C[n]\)\;
			}
		}

		\BlankLine
		\(R.\setRemove{c}\)\;
	}
}

\ifstandalone
\end{algorithm}
\RestoreCaptionOfAlgo
\fi

\end{document}

% \end{algorithm}
%
% \begin{figure}[H]
% 	\centering
% 	\includegraphics[width=\textwidth]{branchAndBound-2}
% 	\caption{inserisci didascalia}
% \end{figure}
%
% \clearpage
% \section{Algoritmi euristici}
%
% Si può ricorrere ad algoritmi \enquote{euristici} che forniscono una soluzione ammissibile, non necessariamente ottima né approssimata.
% Possiamo utilizzare le tecniche di programmazione per algoritmi ingordi o di ricerca locale.
%
% \paragraph{Shortest edges first}
% Cambiamo approccio per il problema del commesso viaggiatore.
% Ordiniamo gli archi per pesi non decrescenti e aggiungiamo archi alla soluzioni seguendo questo ordine finché non sono stati aggiunti \(n-1\) archi, dove \(n\) è il numero di nodi.
% Però, attenzione, per poter aggiungere un arco, occorre verificare che:
% \begin{enumerate*}[label={\arabic*)}]
% 	\item per ciascuno dei suoi nodi non siano stati già scelti due archi;
%     \item che non si formino circuiti (\mfSet);
% \end{enumerate*}
% A questo punto, si è trovata una catena Hamiltoniana e si chiude il circuito aggiungendo l'arco tra i due nodi estremi della catena.
%
% % TODO figure
%
% \begin{algorithm}[H]
% 	\caption{Approccio ingordo al problema del commesso viaggiatore}
% 	%&../preamble

% arara: pdflatex: { synctex: no }
% arara: latexmk: { clean: partial }
\ifstandalone
\begin{document}
\begin{algorithm}[H]
\fi

\prototype{\Set \greedyTsp{\Graph \(G\)}}{

	\BlankLine
	\tcp{creazione strutture dati}
	\Set \(result \Assign  \setConstructor\)\;
	\mfSet \(M \Assign  \mfConstructor{G.\setSize}\)\;
	\tcp{archi in ingresso ad un nodo}
	\Array{\Int} \(edges\) \Assign \new \Array{\Int}[1][G.\setSize] = \{ 0 \} \tcp{no.\ archi della catena}

	\BlankLine
	\(A\) \Assign \{ ordina gli archi per peso decrescente \}

	\BlankLine
	\tcp{per ogni arco che appartiene all'insieme degli archi}
	\ForEach{\((u,v) \in A\)}{
		\tcp{se non si è formato un ciclo}
		\If{\(edges[u] < 2\) \And \(edges[v] < 2\) \And \(M.\mfFind{u} \Neq M.\mfFind{v}\)}{

			\BlankLine
			\tcp{prendo nota dell'arco inserito}
			\(S.\setInsert{\((u,v)\)}\)\;

			\BlankLine
			\tcp{aggiorno il no.\ di lati entranti nei due nodi}
			\(\Increment{edges[u]}\)\;
			\(\Increment{edges[v]}\)\;

			\BlankLine
			\tcp{li considero come un unico nodo}
			\(M.\mfMerge{u,v}\)\;
		}
	}

	\BlankLine
	\tcp{chiudo il circuito}
	\Int \(u \Assign  1\)\;
	\lWhile{\(edges[1] \Neq 1\)}{
		\(\Increment{u}\)
	}

	\BlankLine
	\Int \(v \Assign  u + 1\)\;
	\lWhile{\Not \(edges[v] \Neq 1\)}{
		\(\Increment{v}\)
	}

	\BlankLine
	\tcp{chiusura del circuito hamiltoniano}
	\(S.\setInsert{\((u,v)\)}\)

	\BlankLine
	\tcp{restituisco l'insieme che archi che costituisce il percorso}
	\Return \(S\)\;
}

\ifstandalone
\end{algorithm}
\end{document}
\fi

% \end{algorithm}
%
% \subparagraph{Analisi della complessità}
% L'algoritmo costa \(\Omicron(n^2 \log n)\) a causa dell'\emph{ordinamento degli archi}.
%
% La soluzione ottenuta può essere la base di partenza per un algoritmo \branchAndBound che può essere migliorata tramite ricerca locale.
%
% \paragraph{Approccio Nearest neighbor}
% Si parte da una città e si seleziona come prossima città quella più vicina.
% Si va avanti così, evitando città già visitate.
% Quando si sono visitate tutte le città si torna alla città di partenza.
%
% \subparagraph{Analisi della complessità}
% Questo algortimo ha complessità \(\Omicron(n^2)\) perché per ogni città devo guardare tutte le altre.

% TODO figura

\end{document}

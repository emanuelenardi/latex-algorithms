% \documentclass{article}
\documentclass[answers]{exam}
\renewcommand{\solutiontitle}{\noindent\textbf{Risposta:}\par\noindent}
\SolutionEmphasis{\itshape\small}

% NOTE non ho utlizzato il prembolo comune a tutti i capitoli
% \usepackage{../../00-main.preamble}

\usepackage[utf8]{inputenc}
\usepackage[T1]{fontenc}
\usepackage[italian]{babel}
\usepackage[useregional]{datetime2}

% NOTE ridefinisco i comandi
\title{Domande di ripasso}
\author{Emanuele Nardi}

% arara: pdflatex: { shell: yes, draft: yes }
% arara: pdflatex: { shell: yes, synctex: yes }
% arara: latexmk:  { clean: partial }
\begin{document}
\maketitle

% \begin{abstract}\noindent
% Ho scritto le domande che mi sono servite per fare un ripasso della materia.
% \end{abstract}

\section{Introduzione}

\begin{questions}
	\question Cos'è un problema computazionale?

	\begin{solution}
		Dati un dominio di input e un dominio di output, un problema computazionale è rappresentato dalla relazione matematica che associa un elemento del dominio di output ad ogni elemento del dominio di input.
	\end{solution}

	\question Cos'è un algoritmo?

	\begin{solution}
		Dato un problema computazionale, un algoritmo è un procedimento effettivo, espresso tramite un insieme di passi elementari ben specificati in un sistema formale di calcolo, che risolve il problema in tempo finito.
	\end{solution}

	\question Cos'è un invariante?

	\begin{solution}
		Condizione sempre vera in un certo punto del programma.
    \end{solution}

	\question Cos'è un invariante di ciclo?
		Cosa ci permette di dimostrare?

	\begin{solution}
		Una condizione sempre vera all’inizio dell’iterazione di un ciclo.
		Il concetto di invariante di ciclo ci aiuta a dimostrare la correttezza di un algoritmo iterativo.
	\end{solution}

	\question Cos'è un invariante di classe?

	\begin{solution}
		Una condizione sempre vera al termine dell’esecuzione di un metodo della classe.
	\end{solution}

	\question Quali sono i criteri per calcolare la dimesione dell'input? Fai un esempio.

	\begin{solution}
		TODO.
	\end{solution}

\end{questions}

\end{document}

%&preamble.main

\setcounter{section}{10}

% NOTE impostazione percorsi di default
% https://tex.stackexchange.com/questions/79058/
\makeatletter
\providecommand*{\input@path}{}
\g@addto@macro\input@path{%
	{assets/algorithms/11/}%
}% append
\makeatother

\graphicspath{
	{assets/figures/11/}%
}

\title{Scelta degli algoritmi}

% arara: pdflatex: { draft: yes, synctex: no }
% arara: pdflatex: { synctex: yes }
% arara: latexmk: { clean: partial }
\begin{document}
% \maketitle

\chapter{Cammini minimi, sorgente singola}

\begin{fquote}
	[Steven S. Skiena]
	[The Algorithm Design Manual]
	[] -v
In algorithms, as in life, persistence usually pays off.
\end{fquote}

\begin{note}
La prima cosa da fare quando si progetta un algoritmo è capire quale sia la struttura dati adatta a risolvere quel particolare problema.
\end{note}

\subsection{Introduzione}

\begin{definition}[costo del cammino]
Dato un cammino \(p = \langle v_1, v_2, \ldots, v_k \rangle\) con \(k > 1\), il \emph{costo del cammino} è dato da
\begin{equation*}
w(p) = \sum_{i=2}^k w(v_{i-1}, v_i)
\end{equation*}
\end{definition}

\paragraph{Definizione del problema}
Dati in input un grafo orientato \(G = (V, E)\), un nodo sorgente \(s\) ed una funzione di peso \(w\colon E \to R\) (che associa ad ogni arco un numero reale che rappresenta il peso).

Trovare un cammino da \(s\) ad \(u\), per ogni nodo \(u \in V\), il cui costo sia minimo, ovvero più piccolo o uguale del costo di qualunque altro cammino da \(s\) a \(u\).

\begin{note}
Non ci limitiamo a trovare un solo percorso, ma tutti i cammini da un nodo a tutti gli altri nodi.
\end{note}

\paragraph{Panoramica sul problema}

Per risolvere il problema del cammino minimo fra una coppia di vertici, si risolve il problema di cammini minimi da sorgente unica (si trovano tutti i cammini che partono da un nodo) e si estrae il cammino richiesto.
Per quanto riguarda il \emph{caso pessimo} non si conoscono algoritmi che abbiamo tempo di esecuzione migliore.

In alcuni casi gli archi possono avere peso negativo. Questo influisce sul problema (se è ben definito oppure no) e sulla soluzione (in assenza di archi negativi si devono utilizzare tecniche diverse).

Nell'algoritmo di Dijkstra si suppone che tutti gli archi abbiano peso positivo, mentre nell'algoritmo di Bellman-Ford gli archi possono avere peso negativo, ma non possono esistere cicli di peso negativo.

\begin{note}
In generale possiamo ammettere pesi negativi, ma non cicli negativi.
\end{note}

\paragraph{Considerazioni sui cicli}
Se esiste un ciclo di peso negativo raggiungibile dalla sorgente, non esistono cammini finiti di peso minimo;
per qualunque cammino, bastera passare per un ciclo negativo più volte per ottenere un ciclo di costo inferiore.

Ovviamente, in un cammino minimo \emph{non è possibile la presenza di un ciclo di peso positivo}. I cicli di peso nullo possono essere banalmente eliminati dal cammino minimo, in quanto inutili e ridondanti.

\subsubsection{Sottostruttura ottima}

\begin{definition}[albero dei cammini minimi]
L'albero dei cammini minimi è un albero di copertura radicato in \(s\) avente un cammino da \(s\) a tutti i nodi raggiungibili da \(s\).
\end{definition}

\begin{note}
Non bisogna far confusione con gli alberi di copertura di peso minimo.
\end{note}

\paragraph{Soluzione ammissibile}
Una soluzione \emph{ammissibile} può essere descritta da un \emph{albero di copertura \(T\)} radicato in \(s\) e da un \emph{vettore delle distanze \(d\)}, i cui valori \(d[u]\) rappresentano il costo del cammino da \(s\) a \(u\) in \(T\).

\tikzset{
	,thick
	,font = \ttfamily\bfseries\small
	,mynode/.style  = {circle, draw=black, align=center, fill=none}
	,mynoder/.style = {circle, draw=black, align=center, fill=red!40}
	,edgen/.style = {-}
	,edger/.style = {->, thick, blue}
}
\begin{figure}[H]
\begin{subfigure}[t]{.5\linewidth}\centering
	\begin{tikzpicture}
		\node[mynoder, label={above: \(d[A]=0\) }] at (0.0, 2.0) (a) {A};
		\node[mynode,  label={above: \(d[B]=3\) }] at (2.0, 2.0) (b) {B};
		\node[mynode,  label={below: \(d[C]=4\) }] at (2.0, 0.0) (c) {C};
		\node[mynode,  label={below: \(d[D]=6\) }] at (0.0, 0.0) (d) {D};
		\draw[edger] (a) edge node[above] {3} (b);
		\draw[edger] (b) edge node[right] {1} (c);
		\draw[edger] (c) edge node[below] {2} (d);
		\draw[edgen] (a) edge node[left]  {3} (d);
	\end{tikzpicture}
	\caption{Soluzione ammissibile (albero di peso minimo)}%
	\label{fig:etichetta}
\end{subfigure}
\begin{subfigure}[t]{.5\linewidth}\centering
	\begin{tikzpicture}
		\node[mynoder, label={above: \(d[A]=0\) }] at (0.0, 2.0) (a) {A};
		\node[mynode,  label={above: \(d[B]=3\) }] at (2.0, 2.0) (b) {B};
		\node[mynode,  label={below: \(d[C]=4\) }] at (2.0, 0.0) (c) {C};
		\node[mynode,  label={below: \(d[D]=3\) }] at (0.0, 0.0) (d) {D};
		\draw[edger] (a) edge node[above] {3} (b);
		\draw[edger] (b) edge node[right] {1} (c);
		\draw[edgen] (c) edge node[below] {2} (d);
		\draw[edger] (a) edge node[left] {3} (d);
	\end{tikzpicture}
	\caption{Soluzione ottima (albero dei cammini minimi)}%
	\label{fig:etichetta}
\end{subfigure}
\caption{Nota la differenza fra i due alberi: a sinistra l'albero dei \emph{cammini minimi} che rappresenta una soluzione ammissibile, ma non ottima, mentre a destra l'albero dei cammini minimi, ossia l'insieme dei percorsi che minimizzano il peso fra il nodo sorgente e tutti gli altri nodi e rappresenta la soluzione ottima}
\end{figure}
% \includestandalone[mode=image]{shortestPath_ammissibile}

\begin{note}
In questo problema devo trovare i percorsi che minimizzano il peso fra un nodo e tutti gli altri nodi, e \textbf{non}, come sembra spontaneo fare, l'albero che ha complessivamente peso minimo.
\end{note}

\paragraph{Rappresentazione dell'albero}
Per rappresentare l'albero, utilizziamo la rappresentazione basata su vettore dei padri, così come abbiamo fatto con le visite in ampiezza/profondità.

\subsubsection{Teorema di Bellman}

\begin{theorem}[Teorema di Bellman]
Una soluzione ammissibile \(T\) è (anche) ottima se e solo se:
\begin{align*}
d[v] \bs{=} d[u] + w (u,v)			&& \textrm{per ogni arco \((u,v) \bs{\in T}\)} \\
d[v] \bs{\leqslant} d[u] + w (u,v)	&& \textrm{per ogni arco \((u,v) \bs{\in E}\)}
\end{align*}
\end{theorem}

\begin{proof}[Dimostrazione per assurdo (parte 1)]
Sia \(T\) una soluzione ottima.
Consideriamo un qualunque arco \((u,v) \in E\) e sia \(w (u, w)\) la sua lunghezza.

Ovviamente se \((u,v) \in T\), allora \(d[v] = d[u] + w(u,v)\).
Invece se \((u,v) \notin T\), allora poiché \(T\) è ottimo, deve risultare \(d[v] \leqslant d[u] + w(u,v)\), altrimenti esisterebbe nel grafo \(G\) un cammino da \(s\) a \(v\) più corto di quello in \(T\), che è \emph{assurdo} perché abbiamo ipotizzato che \(T\) fosse ottima.
\end{proof}

\begin{proof}[Dimostrazione per assurdo (parte 2)]
Supponiamo per assurdo che il cammino da \(s\) a \(u\) in \(T\) non sia ottimo.
Allora esiste un cammino da \(s\) a \(u\) con distanza \(d'[u]<d[u]\).
Sia \(d'[v]\) la distanza da \(s\) ad un generico nodo \(v\) che appare in tale cammino.
Poichè \(d'[s] = d[s] = 0\), ma \(d'[u]<d[u]\), esiste un arco \((h,k)\) per cui \(d'[h] \geqslant d'[h]\) e \(d'[k]< d[k]\).
Per costruzione \(d_h' + w(h,k) = d_k'\).
Per ipotesi \(d_h + w(h,k) \ge d_k\).
Combinando queste due relazioni, si ottiene:
\begin{equation*}
d_k' = d_h' + w(h,k) \geqslant d_h + w(h,k) \geqslant d_k
\end{equation*}
che contraddice l'ipotesi.
\end{proof}

\subsubsection{Verso un algoritmo}

\paragraph{Algoritmo prototipo}
L'algoritmo prende in input un grafo e il nodo sorgente.
I pesi vengono estratti dalla struttura dati \Graph.
Inizializziamo \(d\) con una sovrastima della distanza; \(d[s] = 0\) sta a significare che la sorgente ha distanza da sè stessa pari a 0 (caso base) e con \(d[x] = +\infty\) indico che la distanza di tutti gli altri nodi, fintanto che non è nota, è pari a \(+\infty\).

\begin{algorithm}[H]
	\caption*{Algoritmo generico}
	%&../preamble

% \documentclass{standalone}
% \usepackage{../preamble}

% arara: pdflatex: { synctex: no }
% arara: latexmk: { clean: partial }
\ifstandalone
\begin{document}
\NoCaptionOfAlgo
\begin{algorithm}[H]
\caption[Algoritmo prototipo]{}
\fi

\tcp{Algoritmo prototipo dei cammini minimi}
\prototype{\Int, \Int \CamminiMinimi{\Graph \(G\), \Node \(s\)}}{

	\BlankLine
	\tcp{Inizializza \(T\) ad una foresta di copertura composta da nodi isolati}
	\tcp{Inizializza \(d\) con una sovrastima della distanza (\(d[s] = 0\), \(d[x] = +\infty\))}

	\BlankLine
	\While{\( \exists (u,v) \colon d[u] + G.\weight{u,v} < d[v] \)}{
		\tcp{Esiste un arco che mi permette di migliorare la stima}

		\BlankLine
		\(d[v] = d[u] + \weight{u,v}\) \Comment*[h]{Aggiorno la distanza}\;
		\tcp{Sostituisci il padre di \(v\) in \(T\) con \(u\)}
	}
	\Return \((T,d)\)
}

\ifstandalone
\end{algorithm}
\RestoreCaptionOfAlgo
\end{document}
\fi

\end{algorithm}

\begin{note}
Se al termine dell'esecuzione qualche nodo mantiene una distanza infinita, allora esso non è raggiungibile.
\end{note}

\paragraph{Algoritmo generico}
\Array{\Bool} ci permette di sapere in tempo costante se un certo nodo appartiene ad una struttura dati oppure no, non è necessario quando si implementerà realmente il codice.

\begin{algorithm}[H]
	\caption*{Algoritmo prototipo}
	%&../preamble

% arara: pdflatex: { synctex: no }
% arara: latexmk: { clean: partial }
\ifstandalone
\begin{document}
\begin{algorithm}[H]
\fi

(\Int, \Int)\prototype{ \CamminiMinimi{\Graph \(G\), \Node \(s\)}}{

	\BlankLine
	\tcp{Inizializzazione dei vettori}
	\Array{\Int} \(d\) \Assign \new \Array{\Int}[1][G.n]
	\Comment*[r]{distanze dalla sorgente}

	\Array{\Int} \(T\) \Assign \new \Array{\Int}[1][G.n]
	\Comment*[r]{vettore dei padri}

	\Array{\Bool} \(b\) \Assign \new \Array{\Bool}[1][G.n]
	\Comment*[r]{per sapere in tempo costante se \(u \in S\)}

	\BlankLine
	\tcp{Inizializzo tutti i nodi tranne la sorgente}
	\ForEach{\(u \in G.\VV - \{s\}\)}{
		\(T[u] \Assign \Nil\) \Comment*[l]{non hanno padri}
		\(d[u] \Assign +\infty\) \Comment*[l]{non li ho ancora raggiunti}
		\(b[u] \Assign \False\) \Comment*[l]{non appartengono ancora all'insieme}
	}

	\BlankLine
	\tcp{Inizializzo la sorgente}
	\(T[s] \Assign \Nil\) \Comment*[h]{non ha padre}\;
	\(d[s] \Assign 0\) \Comment*[h]{per convenzione}\;
	\(b[s] \Assign \True\) \Comment*[h]{appartiene all'insieme}\;

	\BlankLine
	\lnl{shortestPath:init}%
	\alert{\DataStructure \(S \Assign \dsConstructor\)}\;
	\alert{\(S\).\dsAdd{s}}\;

	\BlankLine
	\While{\Not \(S.\setEmpty\)}{

		\lnl{shortestPath:remove}%
		\alert{\Int \(u \Assign S.\extract\)} \Comment*[l]{estraggo un nodo}
		\Array{b}{u} \Assign \False \Comment*[l]{non è più contenuto nella struttura dati}

		\BlankLine
		\ForEach(\Comment*[h]{per tutti i vicini}){\(v \in G.\adj(u)\)}{

			\BlankLine
			\If(\Comment*[h]{se migliora la stima}){\(d[u] + G.\weight{u,v} < d[v]\)}{
				% NOTE eIf incompatibile con l'opzione 'onelanguage'
				% NOTE in realtà è la traduzione sbagliata di "Else" in "allora"
				\eIf(\Comment*[h]{se non fa già parte dell'insieme}){\Not \(b[v]\)}{
					\lnl{shortestPath:add}%
					\alert{\(S.\dsAdd(v)\)} \Comment*[h]{aggiungilo}\;
					\(b[v] \Assign \True\) \Comment*[h]{fa parte dell'insieme}\;
				}{
					\lnl{shortestPath:shortest-update}%
					\alert{\footnotesize\ttfamily// Azione da intraprendere nel caso \(v\) sia già presente in \(S\)}
				}

				\BlankLine
				\tcp{aggiorno i vettori}
				\(T[v] \Assign u\)\;
				\(d[v] \Assign d[u] + G.\weight{u,v}\)\;
			}
		}
	}

	\Return \((T,d)\)\;
}

\ifstandalone
\end{algorithm}
\end{document}
\fi

\end{algorithm}

% TODO riprendi correzione da qui
\subsection{Dijkstra}

Il seguente algoritmo è stato sviluppato da Edsger W. Dijkstra nel 1956, pubblicato nel 1959.
Nella versione originale veniva utilizzato per trovare la distanza minima fra due nodi sfruttando il concetto di coda con priorità.
Tieni conto però che le code di priorità basate sugli heap binari sono state proposte nel '64, infatti l'algoritmo che di solito viene considerato di Dijkstra in realtà è la versione modificata di Johnson.

\paragraph{Implementazione}
L'algoritmo utilizza una coda con priorità basata su vettore.

\begin{algorithm}[H]
	\caption{Algoritmo di Dijkstra}
	\setcounter{AlgoLine}{0}
	\documentclass[varwidth=6in]{standalone}
\usepackage{../_preamble}

% arara: pdflatex: { synctex: no }
% arara: latexmk: { clean: partial }
\begin{document}
\ifstandalone
\NoCaptionOfAlgo
\begin{algorithm}[H]
\caption[Algoritmo di Dijkstra]{}
\fi

\prototype{\Array{\Int}, \Array{\Int} \CamminiMinimi{\Graph \(G\), \Node \(s\)}}{

	\BlankLine
	\lnl{dijkstra:init}%
	\alert{ \Heap \(S\) \Assign \heapConstructor } \Comment*[l]{\(\Omicron(n) \cdot 1\)}
	\alert{ \(S\).\heapInsert{\(s\), \(0\)} }\;

	\BlankLine
	\While(\Comment*[h]{\(\Omicron(n)\)}){\Not \(S.\setEmpty{}\)}{

		\lnl{dijkstra:remove}%
		\tcp{\(\Omicron(n)\) vettore ordinato / \(\Omicron(\log n)\) heap binario}
		\alert{\Int \(u\) \Assign \(S\).\heapDeleteMin }\;
		\ArrayCall{b}{u} \Assign \False\;
	%
		\BlankLine
		\ForEach{\( v \in G.\adj{u} \)}{
			\If{\( \ArrayCall{d}{u} + G.\weight{u,v} < \ArrayCall{d}{v} \)}{

				\BlankLine
				\eSea{\(\Not \ArrayCall{b}{v}\)}{
					\lnl{dijkstra:add}%
					\tcp{\(\Omicron(1) \cdot n\) vettore ordinato / \(\Omicron(\log n) \cdot n\) heap binario}
					% \alert{ \(S.\heapInsert{\(v\), \(\ArrayCall{d}{u} + G.\weight{u,v}\)}\)} \;
					\ArrayCall{b}{v} \Assign \True\;
				}{
					\lnl{dijkstra:update}%
					\tcp{\(\Omicron(1) \cdot m\) vettore ordinato / \(\Omicron(\log n) \cdot m\) heap binario}
					% \alert{ \(S\).\heapDecrease{\(v\), \(\ArrayCall{d}{u} + G.\weight{u,v}\)} }\;
				}
				\ArrayCall{T}{v} \Assign \(u\)\;
				\ArrayCall{d}{v} \Assign \(\ArrayCall{d}{u} + G.\weight{u,v}\)\;
			}
		}
	}
	\Return \((T, d)\)\;
}

\ifstandalone
\end{algorithm}
\RestoreCaptionOfAlgo
\fi
\end{document}

\end{algorithm}

\paragraph{Commento sulla complessità}

\begin{itemize}
	\item[\circled{\ref{dijkstra:init}}]
	Viene creato un vettore di dimensione \(n\).
	Ogni elemento \(u\)-esimo rappresenta il nodo \(u\).
	Le priorità (distanze) vengono inizializzate ad \(+\infty\).
	La priorità di \(s\) è posta uguale a \(0\).
	Per un costo computazionale di \(O(n)\);

	\item[\circled{\ref{dijkstra:remove}}]
	Si ricerca il minimo all'interno del vettore, una volta trovato si \enquote{cancella} la sua priorità.
	Per un costo computazionale di \(O(n)\);

	\item[\circled{\ref{dijkstra:add}}]
	Si registra la priorità nella posizione corrispondente all'indice \(v\).
	Per un costo computazionale di \(O(1)\);

	\item[\circled{\ref{dijkstra:update}}]
	Si aggiorna la priorità nella posizione corrispondente all'indice \(v\).
	Per un costo computazionale di \(O(1)\).
\end{itemize}

\paragraph{Commento}
Tutte le volte che estraiamo un nodo, quel nodo ha una distanza (priorità) positiva ed estraiamo nodi a distanza progressivamente crescenti.
Se estraggo un nodo dalla coda tutti gli altri nodi hanno distanze più grandi.
Tutte le volte che estraggo un nodo la sua distanza non può più essere modificata.
Ed è questo il motivo per cui l'algoritmo di Dijkstra funziona bene solo con pesi positivi.

\begin{note}
L'algoritmo di Dijkstra funziona bene solo con pesi positivi.
\end{note}

\subsubsection{Correttezza per pesi positivi}

Ogni nodo viene estratto una e una sola volta.
Al momento dell’estrazione la sua distanza è minima.

\begin{proof}[Dimostrazione per induzione sul numero \(k\) di nodi estratti]
Per \(k=0\) (caso base) è vero poiché \mbox{\(d[s]=0\)} e non ci sono lunghezze negative.
Supponiamo che sia vero per i primi \(k-1\) nodi (ipotesi induttiva).
Quando viene estratto il \(k\)-esimo nodo \(u\), la sua distanza \(d[u]\) dipende dai \(k-1\) nodi già estratti (passo induttivo).
Non può quindi dipendere dai nodi ancora da estrarre, che hanno distanza \(\geqslant d[u]\).
Di conseguenza \(d[u]\) è minimo e \(u\) non verrà più re-inserito, perché non ci sono distanze negative.
\end{proof}

\subsection{Johnson}

\paragraph{Analisi della complessità}
\(\Omicron(n^2 + m)\) ma siccome \(m = \Omicron(n^2)\), allora il costo è \(\Omicron(n^2)\), con l'introduzione dell'heap binario nel '64 le operazioni che prima venivano svolte con complessità \(\Omicron(n)\) sul vettore ordinato passano ad avere una complessità di \(\Omicron(\log n)\).
Di conseguenza la complessità dell'algoritmo diventa \(\Omicron(m \log n)\).

\smallskip
Per \emph{grafi densi} non conviene utilizzare uno heap binario in quanto \(m = \Theta(n^2)\) e di conseguenza l'algoritmo avrebbe una complessità di \(\Omicron(n^2 \log n)\), mentre per \emph{grafi sparsi} \(m = \Theta(n)\) e l'algoritmo \(\Omicron(n \log n)\).

\subsection{Fredman-Tarjan}

\paragraph{Analisi della complessità}
Sfruttando un heap di fibonacci l'operazione di \heapDecrease ha costo ammortizzato costante; così facendo hanno abbassato la complessità a \(\Omicron(m + n \log n)\).
Per \emph{grafi sparsi} produce un miglioramento nella complessità.

\subsection{Bellman-Ford-Moore}

\begin{note}
\'E computazionalmente più pesante dell'algoritmo di Dijkstra ma può lavorare anche con archi di peso negativo.
\end{note}

\paragraph{Implementazione}
Coda senza priorità.

\begin{algorithm}[H]
	\caption{Algoritmo di Bellman-Ford-Moore}
	%&../preamble

% \documentclass[varwidth=6in]{standalone}
% \usepackage{../preamble}

% arara: pdflatex: { synctex: no }
% arara: latexmk: { clean: partial }
\ifstandalone
\begin{document}
\NoCaptionOfAlgo
\begin{algorithm}[H]
\caption[Bellman-Ford-Moore]{}
\fi

\prototype{\Array{\Int}, \Array{\Int} \CamminiMinimi{\Graph \(G\), \Node \(s\)}}{

	\BlankLine
	\lnl{bellman:init}%
	\alert{ \Queue \(S\) \Assign \queueConstructor }\;
	\alert{ \(S.\queueInsert{s}\) } \Comment*[h]{metto in coda la sorgente}\;

	\BlankLine
	\While(\Comment*[h]{\(\Omicron(n)\)}){\Not \(S\).\setEmpty}{

		\lnl{bellman:remove}%
		\alert{ \Int \(u\) \Assign \(S\).\queueRemove } \Comment*[l]{\(\Omicron(1 \cdot n)\)}
		\ArrayCall{b}{u} \Assign \False\;

		\BlankLine
		\ForEach{\( v \in G.\adj{u} \)}{
			\If{\( \ArrayCall{d}{u} + G.\weight{u,v} < \ArrayCall{d}{v} \)}{

				\BlankLine
				\If{\Not \ArrayCall{b}{v}}{
					\lnl{bellman:add}%
					\tcp{lo metto in cosa quando c'è un miglioramento}
					\alert{ \(S.\queueInsert{v}\) } \Comment*[h]{\(\Omicron(m \cdot n)\)}\;
					\ArrayCall{b}{v} \Assign \True\;
				}

				\ArrayCall{T}{v} \Assign \(u\)\;
				\ArrayCall{d}{v} \Assign \(\ArrayCall{d}{u} + G.\weight{u,v}\)\;
			}
		}
	}
	\Return \((T, d)\)\;
}

\ifstandalone
\end{algorithm}
\RestoreCaptionOfAlgo
\end{document}
\fi

\end{algorithm}

\begin{itemize}
	\item [\circled{\ref{bellman:init}}]
	Viene creata una coda di dimensione \(n\).
	Per un costo computazionale di \(O(n)\);
	\item [\circled{\ref{bellman:remove}}]
	Viene estratto il prossimo elemento della coda.
	Per un costo computazionale di \(O(1)\);
	\item [\circled{\ref{bellman:add}}]
	Si inserisce l'indice \(v\) in coda.
	Per un costo computazionale di \(O(1)\).
\end{itemize}

L'inserimento in coda può essere fatto più di una volta durante il ciclo di esecuzione dell'algoritmo, al contrario di quel che accade nell'algoritmo di Dijkstra.
Il passo \circled{4} non è necessario in quanto non c'è una priorità da aggiornare.

Il vettore delle distanze è rappresentato dalle colonne, su fondo potete vedere lo stato della coda.

\paragraph{Dimostrazione di correttezza}

\paragraph{Analisi di complessità}

\paragraph{Cammini minimi su DAG}

I cammini minimi su DAG sono sempre ben definiti;
anche in presenza di pesi negativi, in quanto non esistono cicli (nè tantomeno quelli negativi).
\'E possibile rilassare gli archi \emph{in ordine topologico}, \emph{una volta sola}.
Non essendoci cicli, non c'è modo di tornare su un nodo già visitato ed abbassare il valore della sua distanza (il suo campo \(d\)).

Si utilizza quindi l'ordine topologico.

% TODO scrivere algoritmo
% \includestandalone{bellman-dag}

\subsubsection{Riassumendo}

\begin{table}[H]\centering
	\caption{Quale complessità preferire?}%
	\label{tab:complexity-compared}
	\begin{tabular}{@{} *{3}{l} @{}}\toprule
		\textbf{Algoritmo}		& \textbf{Complessità}				& \textbf{Input}  \\\midrule
		Dijkstra		& \(\Omicron(n^2)\)				& Pesi positivi, grafi denso  \\\lightrule
		Johnson			& \(\Omicron(m \log n)\)		& Pesi positivi, grafi sparso \\\lightrule
		Fredman-Tarjan	& \(\Omicron(m + n \log n)\)	& Pesi positivi, grafi denso, dimensioni molto grandi \\\lightrule
		Bellman-Ford	& \(\Omicron(m \cdot n)\)		& Pesi negativi \\
						& \(\Omicron(m + n)\)			& DAG			\\\lightrule
		BFS				& \(\Omicron(m + n)\)			& Senza pesi	\\\bottomrule
	\end{tabular}
\end{table}

\subsubsection{Cammini minimi, sorgente multipla}

Vogliamo cercare i cammini minimi fra tutti i nodi.

\begin{table}[H]\centering
	\caption{Quale complessità preferire?}%
	\label{tab:complexity-compared}
	\begin{tabular}{@{} *{3}{l} @{}}\toprule
		\textbf{Algoritmo}		& \textbf{Complessità}				& \textbf{Input}  \\\midrule
		\multirow{2}*{\parbox{2cm}{Pesi positivi, grafo denso}}		& \(\Omicron(n \cdot n^2)\)				& Applicazione ripetuta (\(n\)) dell'algoritmo di Dijkstra \\
		&&\\\lightrule
		Pesi positivi, grafo sparso & \(O(n \cdot (m \log n))\) & Applicazione ripetuta dell'algoritmo di Johnson \\\lightrule
		Pesi negativi & \(O(n \cdot nm)\) & Applicazione ripetuta di Bellman-Ford, \emph{sconsigliata} \\\lightrule
		Pesi negativi, grafo denso & \(O(n^3)\) & Algoritmo di \emph{Floyd e Warshall} \\\lightrule
		Pesi negativi, grafo sparso & \(O(nm \log n)\) & Algoritmo di \emph{Johnson per sorgente multipla}\\\bottomrule
	\end{tabular}
\end{table}

L'algoritmo di Bellman-Ford è sconsigliato per grafi densi perché può arrivare ad avere una complessità di \(\Omicron(n^4)\), mentre l'algoritmo di Floyd e Warshall ha una complessità di \(\Omicron(n^3)\) indipendentemente dalla forma del grafo.

\subsection{Floyd-Warshall}

Utilizza la programmazione dinamica.
Ci riesce ridefinendo la definione del costo di cammino in modo tale che possa essere calcolato in modo ricorsivo.

\end{document}

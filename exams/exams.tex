\documentclass[fleqn]{article}
\usepackage{../00-main.preamble}
\usepackage{../macros}
% \standalonetrue

\title{Eserciziario di algoritmi}
\renewcommand\docversion{v1.0.0}

% NOTE: ciclo di compilazione
% arara: pdflatex: { shell: yes, draft: yes }
% arara: pdflatex: { shell: yes, synctex: yes }
% arara: latexmk:  { clean: partial }
\begin{document}
\maketitle

\part{Esercizi d'esame}

\section*{Premessa}
Ho raccolto \emph{tutti} gli esercizi d'esame dal \(2011\) al \(2019\) riportando fedelmente la consegna.
A margine è riporta la data dell'esame dalla quale l'esercizio è stato estratto, come il numero dello stesso.
Enjoy!

% NOTE tex.stackexchange.com/questions/296070/
\everymath{\textstyle}
\let\displaystyle\textstyle
% \DTMlangsetup{showdayofmonth=false}
\section{Analisi della complessità}

% \begin{minipage}{.7\linewidth}

\esercizio[1]{2014-10-31}
Si considerino le seguenti equazioni di ricorrenza, per le quali i casi base sono tutti pari a \(\T{n} = 1\) per \(n \leqslant 1\).
\begin{itemize}
	\item \(\T{n} = \T{2 \frac{n}{3}} + 2n - 4\)
	\item \(\T{n} = 4 \T{\frac{n}{2}} + n^2 \sqrt{n}\)
	\item \(\T{n} = 2 \T{\frac{n}{4}} + \sqrt{n} + 10 \log n\)
	\item \(\T{n} = 3 \T{\frac{n}{2}} + 2n \log n + 10n\)
	\item \(\T{n} = \T{n-6} + n^{\frac{5}{6}}\)
\end{itemize}
Identificare limiti superiori e inferiori per ognuna delle equazioni di ricorrenza (eventualmente stretti, utilizzando la notazione \(\Theta(f(n))\)), utilizzando un metodo a vostro piacimento.
Assumendo che esse provengano dall’analisi di altrettanti algoritmi, quale algoritmo
scegliereste?

\medskip
\esercizio[1]{2013-01-28}
Supponendo che il caso base sia \(\Omicron(1)\) si calcoli l’andamento asintotico delle seguenti equazioni di ricorrenza:
\begin{itemize}
	\item \(A(n) = 4A\big(\frac{n}{2}\big) + n^2 \log n\)
	\item \(B(n) = 4B\big(\frac{n}{2}\big) + n^2\)
	\item \(C(n) = nC(n-1)\)
\end{itemize}

\medskip
Trovare un limite superiore e inferiore per la seguente ricorrenza:
%
\[
T(n) =
\esercizio[3]{2011-04-11}
\begin{dcases*}
	1 & se \(n = 1\) \\
	\T{\frac{n}{2}} + 1	& se \(n > 1\) è pari \\
	\T{n-2} + 1			& se \(n > 1\) è dispari \\
\end{dcases*}
\]
\begin{hint}
utilizzare i teoremi per avere un'idea della soluzione, ma poi sarà necessario utilizzare il metodo di sostituzione per una dimostrazione formale.
\end{hint}

Trovare un limite superiore e inferiore per la seguente ricorrenza, utilizzando il metodo di sostituzione:
%
\[
T(n) =
\esercizio[4]{2013-01-07}
\begin{dcases*}
	1 & se \(n = 1\) \\
	\T{\frac{n}{4}} + 1	& se \(n > 1\) è pari \\
	\T{n-4} + 1			& se \(n > 1\) è dispari \\
\end{dcases*}
\]

Trovare \emph{limiti superiori e inferiori} per la seguente equazione di ricorrenza, utilizzando il metodo di sostituzione (detto anche per tentativi).
%
\begin{ricorrenza}*[2]{2011-05-02}
	3\T{\frac{n}{4}} + \T{\frac{n}{5}} + n
\end{ricorrenza}
%
\begin{ricorrenza}*[1]{2014-02-03}
	\big( \sum_{i=1}^{\floor{\log n}} \T{\frac{n}{2^i}} \big) + 1
\end{ricorrenza}
%
\begin{ricorrenza}*[1]{2011-05-02}
	3\T{\frac{n}{4}} + \T{\frac{n}{5}} + n
\end{ricorrenza}
%
\begin{ricorrenza}*[1]{2011-06-06}
	\frac{11}{5}n + \T*{\frac{n}{5}} + \T*{\frac{7}{10} n}
\end{ricorrenza}
%
\begin{ricorrenza}*[1]{2015-01-12}
	\T*{\frac{n}{2}} + 2^n
\end{ricorrenza}
%
\begin{ricorrenza}*[1]{2015-02-02}
	\T*{\frac{n}{\sqrt{2}}} + \T*{\frac{n}{\sqrt{4}}} + \T*{\frac{n}{\sqrt{8}}} + \T*{\frac{n}{\sqrt{16}}} + n^2
\end{ricorrenza}
%
\begin{ricorrenza}*[1]{2015-06-08}
	\T*{\frac{n}{\sqrt[3]{2}}} + \T*{\frac{n}{\sqrt[3]{5}}} + \T*{\frac{n}{\sqrt[3]{7}}} + n^3
\end{ricorrenza}
%
\begin{ricorrenza}*[2a]{2015-11-05}
	\T*{\frac{1}{2} n} + \T*{\frac{4}{5} n} + \T*{\frac{3}{10} n} + n^2
\end{ricorrenza}
%
\begin{ricorrenza}*[2b]{2015-11-05}
	\T*{\frac{1}{2} n} + \T*{\frac{2}{5} n} + \T*{\frac{7}{10} n} + n^2
\end{ricorrenza}

Utilizzando un \emph{qualunque metodo}, trovare i limiti inferiori e superiori per le seguenti ricorrenze (assumendo che \(0 < \beta < 1\))
%
\begin{ricorrenza}*[1a]{2015-11-05}
	\T{\beta n} + n^{\beta}
\end{ricorrenza}
%
\begin{ricorrenza}*[1b]{2015-11-05}
	\floor{\frac{1}{\beta}} \T{\floor{\beta n}} + n^{\beta}
\end{ricorrenza}

Trovare \emph{limiti superiori e inferiori} per le seguenti equazioni di ricorrenza:
%
\begin{ricorrenza}*[1]{2011-09-07}
	4 \T{\sqrt{n}} + \log^2 n
\end{ricorrenza}
%
\begin{ricorrenza}*[1]{2011-01-12}
	\T*{\frac{n}{c}} + \Theta(1)
\end{ricorrenza}
%
\begin{ricorrenza}*[1]{2016-01-26}
	\frac{1}{2} \left(\T{n-1} + \T{\frac{3}{4}n}\right) + n
\end{ricorrenza}
\begin{ricorrenza}*
	\T{\frac{1}{2} n} + \T{\frac{1}{4} n} + \T{\frac{1}{6} n} + \T{\frac{1}{12} n} + 1
\end{ricorrenza}
%
\begin{ricorrenza}*[1]{2016-06-06}
	\T{\frac{1}{10} n} + \T{\frac{5}{6} n} + \T{\frac{1}{16} n} + n
\end{ricorrenza}

Trovate il \emph{limite inferiore} per la sequente equazione di ricorrenza:
%
\begin{ricorrenza}*[1]{2016-07-06}
	2T\big(\frac{n}{\sqrt{2}} - 5\big) + n^{\frac{\pi}{2}}
\end{ricorrenza}
%
\begin{ricorrenza}*[1]{2016-08-29}
	\T*{\frac{n}{15}} + \T*{\frac{n}{10}} + 2\T*{\frac{n}{6}} + \sqrt{n}
\end{ricorrenza}

Trovate il \emph{limite superiore} per la sequente equazione di ricorrenza:, utilizzando il metodo di sostituzione (detto anche per tentativi), facendo particolare attenzione ai casi base.
%
\begin{ricorrenza}*[1]{2011-07-18}
	\sqrt{n} \T{\sqrt{n}} + \sqrt{n}
\end{ricorrenza}
%
\begin{ricorrenza}*[1a]{2016-11-03}
	27\T*{\frac{n}{9}} + n\sqrt{n}
\end{ricorrenza}

Trovate il \emph{limite inferiore} per la sequente equazione di ricorrenza:, utilizzando il metodo di sostituzione (detto anche per tentativi), facendo particolare attenzione ai casi base.
%
\begin{ricorrenza}*[1b]{2016-11-03}
	64\T*{\frac{n}{16}} + n\sqrt{n}
\end{ricorrenza}

Trovare \emph{un limite superiore ed un limite inferiore}, i più stretti possibili, per la seguente equazione di ricorrenza, utilizzando il metodo di sostituzione.
%
\begin{ricorrenza}*[1]{2013-07-22}
	2\T{2 \frac{n}{3}} + n^2
\end{ricorrenza}
%
\begin{ricorrenza}*[1]{2012-05-03}
	2\T*{\frac{n}{4}} + \sqrt{n}
\end{ricorrenza}
%
\begin{ricorrenza}*[1]{2017-02-07}
	2\T{\frac{n}{8}} + \sqrt[3]{n}
\end{ricorrenza}

Trovare i \emph{limiti superiori e inferiori} più stretti possibili per la seguente equazione di ricorrenza:
%
\renewcommand\intervalNumber{12}
\begin{ricorrenza}[1]{2017-09-04}
	3\T{\frac{n}{3}} + 6\T{\frac{n}{6}} + 54\T{\frac{n}{12}} + n^2
\end{ricorrenza}
\renewcommand\intervalNumber{1}

Trovare i \emph{limiti superiori e inferiori} più stretti possibili per la seguente equazione di ricorrenza, utilizzando il metodo di sostituzione:
%
\begin{ricorrenza}*[1]{2018-01-31}
	\T*{\frac{n}{2}} + \T{\frac{n}{3}} + n \log n
\end{ricorrenza}

Si consideri la seguente equazione di ricorrenza, parametrizzata rispetto al valore \(k\):
%
\begin{ricorrenza}*[1]{2018-07-25}
	k^2 \T[k]{\frac{n}{k}} + n^{\frac{k}{2}}
\end{ricorrenza}%
%
Si supponga che esistano tre algoritmi, con complessità \(T_2(n)\), \(T_3(n)\), \(T_4(n)\).
Quale algoritmo scartereste?
Giustificate la risposta.

Trovare i limiti \emph{superiori e inferiori} il più stretti possibili per la seguente equazione di ricorrenza:
%
\renewcommand\intervalNumber{6}
\begin{ricorrenza}[1]{2018-08-21}
	3\T*{\frac{n}{3}} + 4\T*{\frac{n}{4}} + 12\T*{\frac{n}{6}} + n^2
\end{ricorrenza}
\renewcommand\intervalNumber{1}
%
\renewcommand\intervalNumber{9}
\begin{ricorrenza}[1]{2019-01-21}
	4\T*{\frac{n}{4}} + 9\T*{\frac{n}{9}} + n \sqrt{n}
\end{ricorrenza}
\renewcommand\intervalNumber{1}

Si trovino, tramite il \emph{metodo della sostituzione}, un limite superiore ed un limite inferiore per la seguente ricorrenza (\(m\) costante intera positiva):
%
\begin{ricorrenza}*[1]{2000-00-00}
	\T{m} + \T{n - m} + 1
\end{ricorrenza}%
%
Fare particolare attenzione ai casi base.

Trovare un \emph{limite superiore}, il più stretto possibile, per la seguente equazione di ricorrenza, utilizzando il metodo di sostituzione.
%
\begin{ricorrenza}*[1]{2014-06-16}
	2\T*{\frac{n}{8}} + \sqrt[3]{n}
\end{ricorrenza}
%
\begin{ricorrenza}*[1]{2014-07-21}
	\underset{1 \leqslant k \leqslant n-1}{min} \{ T[k] + \T{n-k} \} + 1
\end{ricorrenza}
%
\begin{ricorrenza}*[1]{2014-09-01}
	6\T{\frac{n}{8}} + \T{\frac{n}{4}} + 1
\end{ricorrenza}

\subsection{Trovare la complessità di una procedura}

\esercizio[1]{2013-06-17}
Trovare un limite superiore e un limite inferiore alla complessita della seguente procedura:

\NoCaptionOfAlgo
\begin{algorithm}[H]
\caption[]{}

\tcp{chiamata iniziale}
\fun{\(V, 1, n\)}\;

\BlankLine
\prototype{\fun{\Array{\Int} V, \Int i, \Int j}}{

	\BlankLine
	\If{\(i = j\)}{
		\Return \(1\)\;
	}


	\BlankLine
	\Int \(T = 0\)\;
	\From{\(k=i\) \DownTo \(j\)}{
		\AddTo{T}{\ArrayCall{V}{k}}
	}

	\BlankLine
	\Return \(T + \fun{\(V\), \(i\), \(j-1\)} + \fun{\(V\), \(i\), \(i + \floor{\sqrt{j-i+1}}\)}\)\;
}

\end{algorithm}
\RestoreCaptionOfAlgo

\esercizio[1]{2014-01-07}
Calcolare la complessità della procedura \mistery descritto di seguito:

\NoCaptionOfAlgo
\begin{algorithm}[H]
\caption[]{}

\prototype{\mistery{\Int n}}{
	\Int i, j, s, k\;
	\(s \Assign 0\)\;

	\BlankLine
	\From{\(i \Assign 1\) \DownTo \(n\)}{
		\(j \Assign 1\)\;

		\BlankLine
		\While{\(j < n\)}{
			\(k \Assign 1\)\;
			\While{\(k \leqslant n\)}{
				\(\Increment{s}\)\;
				\(\Multiply{k}{3}\)\;
			}

			\BlankLine
			\(\Multiply{j}{2}\)\;
		}
	}
}

\end{algorithm}
\RestoreCaptionOfAlgo

\esercizio[1]{2014-04-24}
Trovare un \emph{limite superiore} alla complessita della seguente procedura.
La procedura \random{\(n\)} ha complessità \(\Omicron(1)\) e ritorna un intero casuale compreso fra \(0\) e \(n-1\).

\NoCaptionOfAlgo
\begin{algorithm}[H]
\caption[]{}

\prototype{\mistery{\Array{\Int} A, \Int i, \Int j}}{
	\lIf{\(j < i\)}{\Return \(0\)}
	\lIf{\(i=j\)}{\Return \(2 \cdot \ArrayCall{A}{i}\)}

	\BlankLine
	\Int \(n \Assign j-i+1\)\;
	\Int \(sum \Assign 0\)\;
	\Int \(k \Assign \random{\(n+1\)}\)\;
	\From{\(r \Assign 1\) \DownTo \(2^k\)}{
		\AddTo{sum}{\ArrayCall{A}{i + \random{n}}}\;
	}

	\BlankLine
	\Return \(sum + \mistery{\(A\), \(i\), \(\floor{\frac{i+j}{2}}\)} + \mistery{\(A\), \(\floor{\frac{i+j}{2}} + 1\), \(j\)}\)\;
}

\end{algorithm}
\RestoreCaptionOfAlgo

% \end{minipage}

\begin{comment}

\subsection*{giugno 2016}

\begin{itemize}
	\item minColoring;
	\item canItalyWin;
	\item minPalindrome;
	\item printAll;
	\item albero di copertura \(k\)-minimale;
\end{itemize}

\section{Problemi relativi ai grafi}

\begin{minipage}[t]{.5\linewidth}

Problemi relativi ai grafi \emph{non pesati}:
\begin{itemize}
	\item cammino più breve (bfs);
	\item componenti (fortemente) connesse (dfs);
	\item verifica ciclicità (dfs);
	\item ordinamento topologico (dfs).
\end{itemize}

\end{minipage}
\begin{minipage}[t]{.5\linewidth}

Problemi relativi ai grafi \emph{pesati}:
\begin{itemize}
	\item cammini peso minimo;
	\item alberi di copertura di peso minimo;
	\item flusso massimo.
\end{itemize}

\end{minipage}

% TODO creare foglio esercizi
\begin{comment}
\includestandalone{../algorithms/06/minDistance}

\begin{algorithm}[H]
\documentclass[varwidth=6in]{standalone}
\usepackage{../_preamble}

% arara: pdflatex: { synctex: no }
% arara: latexmk: { clean: partial }
\begin{document}

\ifstandalone
\NoCaptionOfAlgo
% \begin{minipage}{\textwidth}
\begin{algorithm}[H]
\caption[]{}

\begin{multicols}{2}

\BlankLine
\tcp{determina la distanza \(d(x,y) = \abs{x-y}\) minima fra due elementi di un albero binario di ricerca}
\fi

\prototype{\Int \minDist{\Tree t}}{
	\Tree \(u = t.\minFunction\)\;
	\Int \(\varMin = +\infty\)\;
	\Int \(\varPrev = -\infty\)\;

	\BlankLine
	\While{\(u \Neq \Nil\)}{
		\If{\(u.\varValue - \varPrev < \varMin\)}{
			\(\varMin = u.\varValue - \varPrev\)\;
		}
		\(\varPrev = u.\varValue\)\;
		\(u = u.\succNode\)\;
	}
}
\ifstandalone
\columnbreak

\begin{minipage}{.85\linewidth}

\begin{figure}\centering
	\begin{forest} circled, math tree
		[6
			[4[1][3]]
			[9[,phantom][,phantom]]
		]
	\end{forest}
\end{figure}

Si analizza l'albero seguendo l'ordine crescente (in-visita);
per ogni numero incontrato, si calcola la distanza con il valore precedente e la si confronta con il minimo trovato finora
%  (opportunamente inizializzata a \(+\infty\)).
% Il costo è pari al costo di visita dell'albero, ovvero \(\Omicron(n)\), dove \(n\) è il numero di nodi.

\end{minipage}

\end{multicols}
\fi
\ifstandalone
\end{algorithm}
% \end{minipage}
\RestoreCaptionOfAlgo
\fi
\end{document}

\end{algorithm}

\includestandalone{../algorithms/09/minColoring}

\includestandalone{../algorithms/09/canItalyWin}

\begin{algorithm}[H]
\documentclass[varwidth=6in]{standalone}
\usepackage{../_preamble}

% arara: pdflatex: { synctex: no }
% arara: latexmk: { clean: partial }
\begin{document}

\ifstandalone
\NoCaptionOfAlgo
\begin{minipage}{\linewidth}
\begin{algorithm}[H]
\caption[Ordinamento per far vincere l'Italia ai mondiali]{}

\begin{multicols}{2}

\BlankLine
\tcp{determina un ordinamento in cui l'Italia può vincere i mondiali}
\fi

\prototype{\Bool \canItalyWin{\Graph G, \Int s}}{
	\Array{\Int} \(visited\) \Assign \new \Array{\Int}{1}{G.\setSize}\;

	\BlankLine
	\tcp{inizializzo i nodi come non visitati}
	\ForEach{\(u \in G.\VV\)}{
		\(visited[u] \Assign 0\)\;
	}

	\BlankLine
	\tcp{identifica le componenti connesse}
	\ccdfs{G, d, 1, \(visited\)}\;

	\BlankLine
	\Return \(\sumFunction{\(visited\), 1, \(G.\setSize\)} \Equal G.\setSize\)\;
}

\ifstandalone
\begin{minipage}{.85\linewidth}
\begin{figure}\centering
	% TODO disegnare ordine topologico
\end{figure}
\columnbreak
\end{minipage}
\fi

\prototype{\printOrder{\Graph G, \Int s}}{
	\Array{\Int} \(visited\) \Assign \new \Array{\Int}{1}{G.\setSize}\;

	\tcp{inizializzo i nodi come falsi}
	\ForEach{\(u \in G.\VV\)}{
		\(visited[u] \Assign 0\)\;
	}

	\Return \(\printOrderRec{\(visited\), 1, \(G.\setSize\)} \Equal G.\setSize\)\;
}

\prototype{\printOrderRec{\Graph G, \Node u, \Array{\Bool} \(visited\)}}{
	\(visited[u] \Assign \True\)\;

	\BlankLine
	\ForEach{\(v \in G.\adj{u}\)}{
		\printOrderRec{G, v, \(visited\)}\;
		\Print \((u,v)\)\;
	}
}

\ifstandalone
\end{multicols}
\end{algorithm}
\end{minipage}
\RestoreCaptionOfAlgo
\fi

\end{document}

\end{algorithm}
\paragraph{Analisi della complessità}
Il costo computazionale è \(\Omicron(m+n)\), corrisponde ad una visita del grafo.

\includestandalone{../algorithms/14/minPalindrome}
\paragraph{Analisi della complessità}
La complessit è pari a \(\Omicron(n^2)\), il tempo necessario per inizializzare e riempire una matrice \(n \times n\).

\section{Algoritmi di ordinamento}

\includestandalone{../algorithms/sorting/countingSort}
\includestandalone{../algorithms/sorting/insertionSort}
\includestandalone{../algorithms/sorting/mergeSort}

\esercizio[1]{2012-02-01}
Si supponga di scrivere una variante di \mergeSort, chiamata \mergeSortFour che, invece di suddividere l’array da ordinare in 2 parti (e ordinarle separatamente), lo suddivide in 4 parti, le ordina ognuna riapplicando \mergeSortFour, e le riunifica usando un’opportuna variante \mergeFour di \merge (la quale, naturalmente, fa la fusione su 4 sottoarray invece di 2).
Come cambia, se cambia, la complessità temporale di \mergeSortFour rispetto a quella di \mergeSort?
Come cambia, se cambia, la complessità temporale di una variante \mergeSortK di \mergeSort che spezza l’array in \(k\) sottoarray?
Giustificare brevemente le risposte.
Non si richiede di scrivere gli algoritmi.

\esercizio[2]{2012-06-18}
Si consideri una versione di MergeSort in cui il vettore viene suddiviso in \(\sqrt{n}\) sottovettori di \(\sqrt{n}\) elementi, ad ognuno dei quali viene applicato \mergeSort in modo ricorsivo, come mostrato nello pseudocodice seguente.

\NoCaptionOfAlgo
\begin{algorithm}[H]
\caption[]{}

\prototype{\mergeSort{\Array{\Int} A, \Int i, \Int j}}{

	\BlankLine
	\If{\(j-i+1\)}{
		\insertionSort{A, i, j}\;
		\Return\;
	}

	\BlankLine
	\Int \(step = \floor{\sqrt{j-i+1}}\)\;
	\Int \(start = i\)\;

	\BlankLine
	\While{\(start \leqslant j\)}{
		\mergeSort{\(A\), \(start\), \minFunction{\(start+step-1\), j}}\;
		\(start = start + step\)\;
	}
}

\end{algorithm}
\RestoreCaptionOfAlgo

La procedura \merge, non mostrata qui per brevità, effettua un’operazione di unione fra i \(\sqrt{n}\) sottovettori, selezionando ad ogni passo il valore minore e avanzando l’indice sul corrispondente sottovettore.
\begin{enumerate}
	\item discutete della complessità della \merge;
	\item scrivete l’equazione di ricorrenza associata a questa versione di \mergeSort;
	\item risolvete l’equazione di ricorrenza ottenuta al punto precedente.
\end{enumerate}

\esercizio[1]{2012-09-10}
Il professor Sortino ha inventato un nuovo algoritmo di ordinamento, che funziona nel modo seguente.
Il vettore di input viene diviso in tre parti, di dimensioni approssimativamente uguali ad un terzo di \(n\).
Dopo di che, vengono ordinati ricorsivamente le prime due parti del vettore (ovvero i primi due terzi), i secondi due terzi, e di nuovo i primi due terzi.


\NoCaptionOfAlgo
\begin{algorithm}[H]
\caption[]{}

\prototype{\sortinoSort{\Array{\Int} A, \Int i, \Int j}}{
	\eSea{\(j - i + 1 \leqslant 6\)}{
		\insertionSort{A, i, j}\;
	}{
		\Int \(s = \ceil{\frac{j-i+1}{3}}\)\;

		\BlankLine
		\sortinoSort{\(A, i, i + 2s - 1\)}\;
		\sortinoSort{\(A, i + s, j\)}\;
		\sortinoSort{\(A, i, i + 2s - 1\)}\;
	}
}

\end{algorithm}
\RestoreCaptionOfAlgo
\begin{enumerate}
	\item qual è la complessità di questo algoritmo?
	\item (difficile, opzionale) dimostrare per induzione che questo algoritmo è corretto.
	Per comodità, assumete pure che tutti i valori siano distinti.
\end{enumerate}

\end{comment}

\end{document}

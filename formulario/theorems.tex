\usepackage{set-utilities}
\usepackage{set-color}
\usepackage{set-math}
\usepackage{set-quote}
\usepackage{set-forest}
\usepackage{set-algorithm2e}
\usepackage{../../../macros}
% \standaloneconfig{}


\newcommand\docversion{v \(\beta\)}

% NOTE: doppia compilazione per tikz
% arara: pdflatex: { synctex: no }
% arara: pdflatex: { synctex: yes }
\begin{document}

\section*{Soluzioni di ricorrenze nell'analisi di algoritmi}

% -------------------------------------------------------------------------------------------------------------------- %
% -------------------------------------------------------------------------------------------------------------------- %
% -------------------------------------------------------------------------------------------------------------------- %
\begin{multicols}{3}
\subsection*{Ricorrenze lineari \emph{con partizione bilanciata}}

\eqmakebox[precondition1][r]{coatanti intere reali} \fbox{\( \mya \geqslant 1, \myb \geqslant 2 \)}

\vspace{.5em}
\noindent
\eqmakebox[precondition1][r]{coatanti reali} \fbox{\( \myc > 0, \mybeta \geqslant 0 \)}
\vfill\null

\subsection*{Teorema dell'esperto}

\eqmakebox[precondition2][r]{qualsiasi} \fbox{\( \mya \geqslant 1, \myb > 1 \)}

\vspace{.5em}
\noindent
\eqmakebox[precondition2][r]{asint. positiva} \fbox{\( \myFunc \)}
\vfill\null

\subsection*{Ricorrenze lineari di ordine costante}

\eqmakebox[precondition3][r]{costanti intere non negative} \fbox{\( a_1, a_2, \dots, a_h \)}

\vspace{.5em}
\noindent
\eqmakebox[precondition3][r]{costanti reali} \fbox{\( \myc > 0, \mybeta \geqslant 0 \)}
\vfill\null

\end{multicols}
% -------------------------------------------------------------------------------------------------------------------- %
% -------------------------------------------------------------------------------------------------------------------- %
% -------------------------------------------------------------------------------------------------------------------- %
\vspace*{-2em}%
\begin{multicols}{3}
\(
	T(\myn) =
	\begin{dcases}
		\displaystyle \mya\, T\left(\frac{n}{\myb}\right) + \myc {\myn}^{\mybeta} & n \geqslant \mym \\
		\Theta(n) & n \leqslant \mym \leqslant h \\
	\end{dcases}
\)
\vfill\null

% -------------------------------------------------------------------------------------------------------------------- %
% -------------------------------------------------------------------------------------------------------------------- %
\(
	T(\myn) =
	\begin{dcases}
		\displaystyle \mya\, T\left(\frac{\myn}{\myb}\right) + \evidenzia{function}{\myFunc} & n > 1 \\
		\Theta(n) 											& n \leqslant 1 \\
	\end{dcases}
\)
\vfill\null

% -------------------------------------------------------------------------------------------------------------------- %
% -------------------------------------------------------------------------------------------------------------------- %
\(
	T(n) =
	\begin{dcases}
		\tikz[%
			baseline,
			remember picture]{
			\node[%
				draw = purple,
				fill = white,
				densely dashed,
				rectangle,
				rounded corners = 8,
				inner sep = 2pt,
				anchor = base] (sommatoria) {
				\( \displaystyle\sum_{1 \leqslant i \leqslant h} {a}_i \)
			};
		}\: T(n - 1) + \myc {\myn}^{\mybeta} & n \geqslant \mym \\
		\Theta(n) & n \leqslant \mym \leqslant h \\
	\end{dcases}
\)

\begin{comment}
\begin{tikzpicture}[remember picture, overlay]
	\node[
		text = coralred,
		above right = 5pt and -10pt of sommatoria,
		inner sep = 1pt] (a) {\(\mya\)};
	\path[->, coralred] (sommatoria.north) edge [bend left = 45] (a.west);
\end{tikzpicture}
\end{comment}
\vfill\null
\end{multicols}
% -------------------------------------------------------------------------------------------------------------------- %
% -------------------------------------------------------------------------------------------------------------------- %
% -------------------------------------------------------------------------------------------------------------------- %
\vspace*{-1.5em}%
\begin{multicols}{3}
Posto \fbox{\( \myalpha = \log_{\myb} \mya \)} , allora:
\vfill\null

% -------------------------------------------------------------------------------------------------------------------- %
% -------------------------------------------------------------------------------------------------------------------- %
Posto \fbox{\( \myalpha = \log_{\myb} \mya \)} , allora:
\vfill\null

% -------------------------------------------------------------------------------------------------------------------- %
% -------------------------------------------------------------------------------------------------------------------- %
Posto \fbox{\(\mya\) = \mySumRef} , allora:
\vfill\null
\end{multicols}
% -------------------------------------------------------------------------------------------------------------------- %
% -------------------------------------------------------------------------------------------------------------------- %
% -------------------------------------------------------------------------------------------------------------------- %
\vspace*{-2em}%
\begin{multicols}{3}
\ifprintmath
\(
	T(\myn) =
	\begin{dcases}
		\Theta({\myn}^{\myalpha})	& \myalpha > \mybeta \\
		\Theta(\myn \log \myn)		& \myalpha = \mybeta \\
		\Theta({\myn}^{\mybeta})	& \myalpha < \mybeta \\
	\end{dcases}
\)
\else
\begin{mylist}
	\item se \fbox{\( \myalpha > \mybeta \)}	\( \longrightarrow \Theta({\myn}^{\myalpha}) \)
	\item se \fbox{\( \myalpha = \mybeta \)}	\( \longrightarrow \Theta(\myn \log \myn) \)
	\item se \fbox{\( \myalpha < \mybeta \)}	\( \longrightarrow \Theta({\myn}^{\mybeta}) \)
\end{mylist}
\fi
\vfill\null

% -------------------------------------------------------------------------------------------------------------------- %
% -------------------------------------------------------------------------------------------------------------------- %
\begingroup
\raggedright
\renewcommand\arraystretch{1.5}%
\setlength\tabcolsep{2pt}%
\begin{tabularx}{\textwidth}{@{} c l @{\hspace{-10pt}}l @{}}
	% ---------------------------------------------------------------------------------------------------------------- %
	\circled{1}  & \eqmakebox[condition][l]{\( \exists\, \myeps > 0 \)}\(: \myFuncRef \in \Omicron(n^{\myalpha \myminus \myeps})\) & %
	\(T(\myn)\) è \(\Theta({\myn}^{\myalpha \myminus \myeps})\) \\
	% ---------------------------------------------------------------------------------------------------------------- %
	\arrayrulecolor{black!30}\cmidrule[\lightrulewidth]{2-3}
	% ---------------------------------------------------------------------------------------------------------------- %
	\circled{2}  & \eqmakebox[condition][l]{}\(\phantom{:\ } \myFuncRef \in \Theta(n^{\myalpha})\) & %
	\(T(\myn)\) è \(\Theta(\myFuncRef \log \myn)\) \\
	% ---------------------------------------------------------------------------------------------------------------- %
	\arrayrulecolor{black!30}\cmidrule[\lightrulewidth]{2-3}
	% ---------------------------------------------------------------------------------------------------------------- %
	\multirow{3}{*}{\circled{3}} & %
	\multirow{3}{*}{\parbox{5cm}{\raggedright
		\eqmakebox[condition][l]{\( \exists\, \myeps > 0 \)}\(: \myFuncRef \in \Omega(n^{\myalpha \myplus \myeps}), \)\\
		\eqmakebox[condition][l]{\( \exists\, \myc		 \)}\(: 0 < \myc < 1, \)\\
		\eqmakebox[condition][l]{\( \exists\, \mym > 0\, \)}\(: a f(\frac{n}{b}) \leqslant c\; \myFuncRef \)}} & %
	\multirow{3}{*}{\(T(\myn)\) è \(\Theta(\myFuncRef)\)} \\
	% ---------------------------------------------------------------------------------------------------------------- %
\end{tabularx}
\endgroup
\vfill\null

% -------------------------------------------------------------------------------------------------------------------- %
% -------------------------------------------------------------------------------------------------------------------- %
\ifprintmath
\(
	T(\myn) =
	\begin{dcases}
		\Theta(n^{\mybeta \myplus 1})		& \mya = 1 \\
		\Theta({\mya}^{n} {n}^{\mybeta})	& \mya \geqslant 2 \\
	\end{dcases}
\)
\else
\begin{mylist}
	\item se \fbox{\( \mya = 1 \)}			\( \longrightarrow T(\myn)\) è \(\Theta(n^{\mybeta \myplus 1}) \)
	\item se \fbox{\( \mya \geqslant 2 \)}	\( \longrightarrow T(\myn)\) è \(\Theta({\mya}^{n} {n}^{\mybeta}) \)
\end{mylist}
\fi
\vfill\null
\end{multicols}
% -------------------------------------------------------------------------------------------------------------------- %
% -------------------------------------------------------------------------------------------------------------------- %
% -------------------------------------------------------------------------------------------------------------------- %

\end{document}

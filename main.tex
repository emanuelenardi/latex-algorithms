%&preamble.main

\makeatletter
\providecommand*{\input@path}{}
\g@addto@macro\input@path{%
	{assets/chapters/}%	capitoli
	{assets/chapters/to-complete/}% capitoli da completare
}% append
\makeatother

% NOTE ciclo di compilazione
% arara: pdflatex: { shell: yes, draft: yes }
% arara: pdflatex: { shell: yes, synctex: yes }
%! arara: latexmk:  { clean: partial }
\begin{document}

\frontmatter

    % NOTE nessun contatore delle pagine
	\pagenumbering{gobble}

    % NOTE frontespizio
    % % NOTE dipendenza pacchetto 'tabularx'
\begin{titlepage}
\begin{center}
	\Logo

	\vspace{2cm}
	\LARGE{\Department\\}

	\vspace{1cm}
	\Large{\Faculty}

	\vspace{2cm}
	\Large\textsc{\What\\}
	\vspace{.5cm}
	{\Huge\textsc{\Subject}\\}
	{\Large\emph{\Subtitle}}

	\vspace{2cm}
	\begin{tabularx}{\textwidth}{ c X c }
		{\LARGE Autore}	&	& {\LARGE Revisori} \\
		\cmidrule(lr){1-1}\cmidrule(lr){3-3}
		\Author	&	& \firstReviserName \\
				&	& \secondReviserName \\
				&	& \thirdReviserName \\
	\end{tabularx}

	\vfill
	\Large{Anno Accademico \academicYear}
\end{center}
\end{titlepage}

	% \cleardoublepage

    % NOTE pagina con la dedica (non numerata)
	% \thispagestyle{empty}

% NOTE spaziatura di un 1/5 della pagina
\vspace*{\stretch{1}}

\begin{flushright}\em
A tutti coloro che lavorano per lasciare il mondo

un po' meglio di come lo hanno trovato.
\end{flushright}

% NOTE spaziatura di un 4/5 della pagina
\vspace*{\stretch{4}}

	% \cleardoublepage
	% \pagestyle{plain}

\mainmatter

	% NOTE disabilita la protrusion localmente
	\microtypesetup{protrusion = false}

	% TODO controlla la profondità della TOC
	% NOTE % chapthers + sections + subsections
	\setcounter{tocdepth}{2}
	% \tableofcontents

	% \listoffigures

	% \listoftables

	% TODO il pacchetto crea problemi con il preambolo, non carica i pacchetti successivi
	% \listoftheorems[ignoreall, show={theorem, definition}] % NOTE si sovrascrivono
	% \renewcommand{\thmtformatoptarg}[1]{ -- #1}
	% \renewcommand{\listtheoremname}{Elenco dei teoremi}
	% \listoftheorems[ignoreall, show={theorem}]
	% \listoftheorems[ignoreall, show={definition}]
	% \listoftheorems[ignoreall, show={note}]

	% TODO risolvere questo problema
	% \listofalgorithms % NOTE non stampa nulla

	% NOTE (ri)abilita la protrusion
	\microtypesetup{protrusion = true}
	\cleardoublepage

	% TODO scrivere il prologo
	% \subfile{00-prologo}

	% \part{Primo semestre}
	%
	% \pagenumbering{arabic}
	% \pagestyle{copyrighted}
	%
	% \subfile{02-analisi}
	% \subfile{03-funzioni}
	% \subfile{04-strutture}
	% \subfile{05-alberi}
	% \subfile{06-abr}
	% \subfile{08-insiemi-dizionari}
	% \subfile{09-grafi}
	% \subfile{12-dividi}
	%
	% \part{Secondo semestre}
	%
	% \pagestyle{revising}

	% TODO da commentare alla revisione finale
	\listofchanges[style=list, title={Modifiche richieste}]

	% \subfile{11-scelta}
	\subfile{13-dp}
	% \subfile{14-greedy}
	% \subfile{15-local}
	% \subfile{16-backtrack}
	% \subfile{17-prob}
	% \subfile{19-approx}

\end{document}

	% \part{Laboratorio}
	% \part{Esercizi}
	% \part{Orale}
	% \part{Esami}

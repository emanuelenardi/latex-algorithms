\documentclass[00-main.tex]{subfiles}
\standalonetrue
\setcounter{section}{8}
\pagestyle{footer}

% arara: pdflatex: { draft: yes, synctex: no }
% arara: pdflatex: { synctex: yes }
% arara: latexmk: { clean: partial }
\begin{document}
% \maketitle

\section{Grafi}

\lipsum[1]

\includestandalone{assets/algorithms/09/graph}

\includestandalone{assets/algorithms/09/visitGraph}

\lipsum[1]

\includestandalone{assets/algorithms/09/bfsProc}

\newpage
\includestandalone{assets/algorithms/09/erdos}

\includestandalone{assets/algorithms/09/printPath}

\includestandalone{assets/algorithms/09/dfsProc}

\paragraph{Analisi della complessità}
Questo algoritmo ha complessità \(\Omicron(n+m)\) con il grafo implementato con liste di adiacenza, e di \(\Omicron(n^2)\) con matrice di adiacenza.

\includestandalone{assets/algorithms/09/dfsStack}

\includestandalone{assets/algorithms/09/ConnectedComponents}

\subsection{Grafi non pesati}

\subsubsection{Applicazioni dfs: grafo \emph{non} orientato aciclico}

\includestandalone{assets/algorithms/09/hasCycle}

\includestandalone{assets/algorithms/09/dfsSchema}

% TODO inserire immagine grafo

% TODO inserire immagine intervalli

\newpage
\begin{definition}[grafo diretti orientati aciclici]
Un grafo è DAG quando
\end{definition}

\includestandalone{assets/algorithms/09/topSort}

\includestandalone{assets/algorithms/09/hasCycleDAG}

% \newpage
\subsection{Componenti fortemente connesse}

\begin{definition}[grafo fortemente connesso]
Un grafo orientato \(G=(V,E)\) è \alert{fortemente connesso} sse ogni suo nodo è raggiungibile da ogni altro suo nodo.
\end{definition}

\begin{definition}[componente fortemente connessa]
Un grafo \(G' = (V',E')\) è una \alert{componente fortemente connessa} di \(G\) sse \(G'\) è un sottografo connesso e massimale di \(G\).
\end{definition}

\subsection{Algoritmo di Korasaju}

\includestandalone{assets/algorithms/09/korasaju}

\end{document}

%&../preamble

% arara: pdflatex: { synctex: no }
% arara: latexmk: { clean: partial }
\ifstandalone
\begin{document}
\begin{algorithm}[H]
\fi

\prototype{\Array{\Int} \restoGreedy{\Array{\Int} \(t\), \Int \(n\), \Int \(R\), \Array{\Int} \(x\)}}{

	\BlankLine
	\Array{\Int} \(x\) \Assign \new \Array{\Int}[1][n] \tcp{memorizza il numero di tagli da restituire}

	\BlankLine
	\{ ordina le monete in modo \emph{decrescente} \}
	\tcp{\(\Omicron(n)\) se già ordinato, \(\Omicron(n \log n)\) altrimenti}

	\BlankLine
	\From(\tcp*[h]{\(\Omicron(n)\)}){\(i \Assign 1\) \DownTo \(n\)} {

		\BlankLine
		\tcp{il numero di monete di taglio massimo}
		\(x[i] \Assign \left\lfloor \dfrac{R}{t[i]} \right\rfloor\)\;

		\BlankLine
		\tcp{calcolo il resto rimanente}
		\(R \Assign R - x[i] \cdot t[i]\)\;
	}

	\BlankLine
	\tcp{restituisce il numero di tagli da restituire}
	\Return \(x\)
}

\ifstandalone
\end{algorithm}
\end{document}
\fi

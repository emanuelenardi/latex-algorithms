%&../preamble

% arara: pdflatex: { synctex: no }
% arara: latexmk: { clean: partial }
\ifstandalone
\begin{document}
\begin{algorithm}[H]
\fi

\prototype{\Array{\Int} \restoDP{\Array{\Int} t, \Int n, \Int R}}{
	\params{resto}{t}[tagli disponibili]
	\params{resto}{n}[numero di monete]
	\params{resto}{R}[il resto da dare]

	\BlankLine
	\BlankLine
	\Array{\Int} \(DP\) \Assign \new \Array{\Int}[0][R] \tcp{valore della soluzione}
	\Array{\Int} \(\mathit{coin}\) \Assign \new \Array{\Int}[0][R] \tcp{monete usate per un specifico valore}
	\(DP[0] \Assign 0\) \Comment*[h]{caso base}\;

	\BlankLine
	\tcp{RIEMPIO LA TABELLA}
	\From{\(i \Assign 1\) \DownTo \(R\)} {
		\(DP[i] \Assign +\infty\) \tcp{valore iniziale}

		\BlankLine
		\BlankLine
		\From{\(j \Assign 1\) \DownTo \(n\)} {
			\If{\(t[j] \leqslant i\) \And \(DP[i-t[j]] + 1 < DP[i]\)}{
				\tcp{aggiorno il valore}

				\BlankLine
				\(DP[i] \Assign DP[i - t[j]] + 1\) \tcp{registro il valore}
				\(\mathit{coin}[i] \Assign j\) \tcp{la moneta da utilizzare quando il taglio è \(i\)}
			}
		}
	}

	\BlankLine
	\tcp{RICOSTRUISCO LA SOLUZIONE}
	\Array{\Int} \(x\) \Assign \new \Array{\Int}[1][n] \Assign \{0\} \tcp{memorizza il numero di tagli da restituire}
	\While(\tcp*[h]{fintanto che ho resto da dare}){\(R > 0\)}{
		\(\Increment{x[\mathit{coin}[R]]}\) \tcp{incremento il numero di monete di quel taglio}
		\(R \Assign R - t[\mathit{coin}[R]]\) \tcp{decremento il resto residuo}
	}

	\BlankLine
	\tcp{restituisce il numero di tagli da restituire}
	\Return \(x\)\;
}

\ifstandalone
\end{algorithm}
\end{document}
\fi

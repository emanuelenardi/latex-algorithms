%&../preamble

% \documentclass[varwidth=6in]{standalone}
% \usepackage{../preamble}

% arara: pdflatex: { synctex: no }
% arara: latexmk: { clean: partial }
\ifstandalone
\begin{document}
\begin{algorithm}[H]
\fi

\prototype{\Tree \huffman{\Array{\Int} \(c\), \Array{\Int} \(f\), \Int \(n\)}}{
	\tcp{\Array{c}{1}{n}: caratteri dell'alfabeto}
	\tcp{\(f[1 \dots n]\): frequenze dei caratteri}
	\tcp{\(n\): dimensione dell'alfabeto}

	\BlankLine
	\Heap \(Q\) \Assign \minHeapConstructor\;

	\BlankLine
	\From(\Comment*[h]{}{\(\Omicron(n)\)}){\(i \Assign 1\) \DownTo \(n\)}{
		\(Q\).\heapInsert{\(f[i]\), \treeConstructor{\(f[i]\), \(c[i]\)}} \Comment*[h]{}{\(\Omicron(\log n)\)}
	}

	\BlankLine
	\From(\Comment*[h]{\(n\): radice}{\(\Omicron(n)\)}){\(i \Assign 1\) \DownTo \(n-1\)}{
		\BlankLine
		\tcp{estraggo i 2 caratteri meno frequenti}
		\(z_1\) \Assign \(Q\).\heapDeleteMin\;
		\(z_2\) \Assign \(Q\).\heapDeleteMin\;

		\BlankLine
		\tcp{Creo un nuovo nodo}
		\(z \Assign \treeConstructor(z_1.f + z_2.f, \Nil)\)\;
		\(z.\varLeft \Assign z_1\)\;
		\(z.\varRight \Assign z_2\)\;

		\BlankLine
		\tcp{Lo inserisco nella coda}
		\(Q\).\heapInsert{z.f, z}\;
	}

	\Return \(Q\).\heapDeleteMin\;
}

\ifstandalone
\end{algorithm}
\end{document}
\fi

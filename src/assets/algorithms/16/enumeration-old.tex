%&../preamble

% arara: pdflatex: { synctex: no }
% arara: latexmk: { clean: partial }
\ifstandalone
\begin{document}
\begin{algorithm}[H]
\fi

\prototype{\Bool \enumeration{\Item{} S, \Int n, \Int i, \dots}}{
	\params{enumeration-old}{S}[vettore contenente le soluzioni parziali \(S[1][i]\)]
	\params{enumeration-old}{n}[il numero massimo di scelte possibili]
	\params{enumeration-old}{i}[indice corrente]
	\params{enumeration-old}{\dots}[parametri addizionali dipendenti dal problema]

	\BlankLine
	\BlankLine
	\Set \(C = \getChoices{S, n, i, \dots}\) \Comment*[h]{Determina \(C\) in funzione di \(S[1][i-1]\)}

	\params{C}[l'insieme dei possibili candidati per estendere la soluzione]

	\BlankLine
	\ForEach(\tcp*[h]{per ogni possibile scelta fra quelle possibili}){\(c \in C\)}{
		\(S[i] \Assign c\) \Comment*[l]{registro la scelta}

		\BlankLine
		\If{\(\isAdmissible{S, n, i}\)}{
			\tcp{\(S[1][i]\) è una soluzione ammissibile}
			\If{\(\processSolution{S, n, i, \dots}\)}{
				\tcp{vogliamo bloccare l'esecuzione alla prima soluzione possibile}
				\Return \True
			}
		}

		\BlankLine
		\tcp{non decido di fermarmi alla prima soluzione ammissibile}
		\If{\(\enumeration{S, n, \(\mhl{i+1}\), \dots}\)}{
			\tcp{effettuo la \(i+1\)-esima scelta}
			\Return \True
		}
	}
	\tcp{non ho trovato la soluzione cercata}
	\Return \False\;
}

\ifstandalone
\end{algorithm}
\end{document}
\fi

%&../preamble

% arara: pdflatex: { synctex: no }
% arara: latexmk: { clean: partial }
\ifstandalone
\begin{document}
\begin{algorithm}[H]
\fi

Un dizionario è una struttura dati \emph{dinamica}, \emph{non lineare} che memorizza una collezione non ordinata di elementi senza valori ripetuti.
Rappresenta il concetto matematico di \emph{relazione univoca} \(R: D \to C\), o associazione chiave-valore.
Dove \(D\) rappresenta il dominio di elementi detti \emph{chiave}, mentre \(C\) rappresenta il codominio degli elementi detti \emph{valori}.
Ogni \emph{valore} può essere associato a più \emph{chiavi}, ma non il contrario.

\BlankLine
\dictionaryConstructor

\BlankLine
\Item \dictLookup{\Item \(k\)} \Comment*[r]{restituisce il valore associato alla chiave \(k\), \Nil altrimenti}
\Item \dictInsert{\Key \(k\), \Item \(v\)} \Comment*[r]{associa il valore \(v\) alla chiave \(k\)}
\dictRemove{\Key \(k\)} \Comment*[r]{rimuove l'associazione della chiave \(k\)}

\ifstandalone
\end{algorithm}
\end{document}
\fi